% resumo em português
\begin{resumo}
\noindent
 
Este trabalho desenvolveu e analisou uma simulação realista das diferentes fases do sono (REM e NREM) em uma Rede Neural Pulsada
(RNP) e sua influência na consolidação e retenção de memórias. Buscou-se entender melhor a natureza dos processos de aprendizado e
memória no cérebro humano, e como esses processos são afetados pelas diferentes fases do sono. Para isso, as características e
propriedades das diferentes fases do sono foram investigadas para fornecer uma base sólida para a simulação. Em seguida, um modelo
apropriado de RNP foi construído para simular a atividade neural durante o sono. Os resultados da simulação com sono foram
analisados e comparados com uma simulação sem sono e não houveram mudanças muito significativas na retenção de memórias, muito
embora houveram fenômenos interessantes e paralelos com o sono biológico observados na simulação. Espera-se que essa pesquisa
possa contribuir para a compreensão dos mecanismos subjacentes aos processos de aprendizado e memória no cérebro humano e possa
ter implicações para os campos da neurociência computacional e inteligência artificial.

\vspace{0.2cm}   
Palavras-chave: Redes Neurais Pulsadas. Sono. Memória. Plasticidade. Neurociência Computacional. 
\end{resumo}
 
\begin{resumo}[Abstract]	
\begin{otherlanguage*}{english}
\noindent 
 
This study developed and analyzed a realistic simulation of the different phases of sleep (REM and NREM) in a Spiking Neural
Network (SNN) and its influence on the consolidation and retention of memories. The aim was to better understand the nature of
learning and memory processes in the human brain, and how these processes are affected by different sleep phases. To this end, the
characteristics and properties of the different sleep phases were investigated to provide a solid foundation for the simulation.
Then, an appropriate SNN model was constructed to simulate neural activity during sleep. The results of the simulation with sleep
were analyzed and compared with a simulation without sleep, and there were no very significant changes in memory retention,
although there were interesting phenomena and parallels with biological sleep observed in the simulation. It is hoped that this
research can contribute to the understanding of the underlying mechanisms of learning and memory processes in the human brain and
may have implications for the fields of computational neuroscience and artificial intelligence.

\vspace{0.2cm}
Keywords: Spiking Neural Networks. Sleep. Memory. Plasticity. Computational Neuroscience.
\end{otherlanguage*}
\end{resumo}
