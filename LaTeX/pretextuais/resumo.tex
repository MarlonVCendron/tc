% resumo em português
\begin{resumo}
\noindent
 
Na interseção entre a neurociência computacional e a Inteligência Artificial (IA), a modelagem de redes neurais biológicas oferece
uma perspectiva diferente para o entendimento do cérebro humano. As Redes Neurais Pulsadas (RNP), destacam-se por sua capacidade
de emular mais fielmente a dinâmica dos neurônios biológicos que as tradicionais Redes Neurais Artificiais (RNA). Este estudo se
concentra em um dos desafios ainda não totalmente resolvidos na neurociência: o papel do sono na consolidação e retenção de
memórias. Embora o foco principal seja o avanço do conhecimento neurocientífico, as descobertas também podem ser úteis para certas
áreas do desenvolvimento de sistemas de IA, particularmente aqueles que buscam emular processos cognitivos humanos. A proposta é
motivada pelo potencial de tais simulações em trazer novas compreensões sobre a neurociência do sono e suas implicações na
aprendizagem e memória. A pesquisa propõe uma investigação sobre como a simulação de fases do sono em RNPs pode influenciar a
estabilidade de assembleias neuronais, unidades fundamentais para o processo de memória. A hipótese é que a simulação das fases do
sono poderia melhorar a capacidade da rede de reter e consolidar memórias, comparando-se o desempenho da rede com e sem a
simulação do sono. A metodologia proposta envolve o desenvolvimento de um modelo de RNP que incorpora simulações das fases de sono
REM e NREM, com base em conhecimentos atuais sobre a neurobiologia do sono, em uma tarefa de memorização de estímulos visuais.
Experimentos foram conduzidos na intenção de avaliar como as diferentes fases do sono, e a atividade neural durante essas fases,
afetam a consolidação de memórias. Durante os experimentos foi observada atividade neuronal que se assemelha a uma espécie de
sonho, embora não tenham sido observadas mudanças significativas na estabilidade das assembleias neuronais. Apesar dos resultados
se mostrarem promissores, é importante avaliá-los com cautela, uma vez que certa instabilidade foi observada durante as simulações
conduzida neste trabalho, indicando a necessidade de maior aprofundamento no estudo para que a essência dos benefícios do sono,
presente em organismos biológicos, possa ser efetivamente observada em simulações computacionais. Conclui-se, portanto, que para
contornar a complexidade da modelagem do sono em sistemas computacionais, é necessário aprofundar a investigação para melhor
compreensão do fenômeno, bem como entender sua relação com os processos de consolidação da memória e dos processos biológicos
subjacentes.

\vspace{0.2cm}   
Palavras-chave: Redes Neurais Pulsadas. Sono. Memória. Plasticidade. Neurociência Computacional. 
\end{resumo}
 
\begin{resumo}[Abstract]	
\begin{otherlanguage*}{english}
\noindent 
 
At the intersection of computational neuroscience and Artificial Intelligence (AI), the modeling of biological neural networks
offers a different perspective for understanding the human brain. Spiking Neural Networks (SNNs) stand out for their ability to
more faithfully emulate the dynamics of biological neurons compared to traditional Artificial Neural Networks (ANNs). This study
focuses on one of the challenges not yet fully resolved in neuroscience: the role of sleep in the consolidation and retention of
memories. Although the main focus is on advancing neuroscientific knowledge, the findings may also be useful for certain areas of
AI system development, particularly those seeking to emulate human cognitive processes. The proposal is motivated by the potential
of such simulations to bring new understandings about the neuroscience of sleep and its implications in learning and memory. The
research proposes an investigation into how the simulation of sleep phases in SNNs can influence the stability of cell
assemblies, fundamental units for the memory process. The hypothesis is that the simulation of sleep phases could improve the
network's ability to retain and consolidate memories, comparing the network's performance with and without the simulation of
sleep. The proposed methodology involves developing an SNN model that incorporates simulations of REM and NREM sleep phases, based
on current knowledge about the neurobiology of sleep, in a task of memorizing visual stimuli. Experiments were conducted with the
intention of evaluating how different sleep phases, and neural activity during these phases, affect memory consolidation. During
the experiments, neuronal activity resembling a kind of dream was observed, although no significant changes in the stability of
the cell assemblies were observed. Although the results are promising, it is important to evaluate them cautiously, as some
instability was observed during the simulations conducted in this work, indicating the need for further study so that the essence
of the benefits of sleep, present in biological organisms, can be effectively observed in computational simulations. Therefore, it
is concluded that to overcome the complexity of modeling sleep in computational systems, further investigation is needed for a
better understanding of the phenomenon, as well as to understand its relationship with memory consolidation processes and the
underlying biological processes.

\vspace{0.2cm}
Keywords: Spiking Neural Networks. Sleep. Memory. Plasticity. Computational Neuroscience.
\end{otherlanguage*}
\end{resumo}
