% resumo em português
\begin{resumo}
\noindent
 
Este trabalho visa desenvolver e analisar uma simulação realista das diferentes fases do sono em uma Rede Neural Pulsada (RNP)
e sua influência na consolidação e retenção de memórias. Busca-se entender melhor a natureza dos processos de aprendizado e
memória no cérebro humano, e como esses processos são afetados pelas diferentes fases do sono. Para isso, as características e
propriedades das diferentes fases do sono serão investigadas para fornecer uma base sólida para a simulação. Em seguida, um modelo
apropriado de RNP será selecionado e utilizado para simular a atividade neural durante o sono. Finalmente, os resultados da
simulação serão analisados para determinar como a consolidação e retenção de memórias são afetadas pelas diferentes fases do sono
e pela atividade neural durante essas fases. Espera-se que essa pesquisa possa contribuir para a compreensão dos mecanismos
subjacentes aos processos de aprendizado e memória no cérebro humano e possa ter implicações significativas para os campos da
neurociência computacional e inteligência artificial.

\vspace{0.2cm}   
Palavras-chave: Redes Neurais Pulsadas. Sono. Memória. Plasticidade. Neurociência Computacional. 
\end{resumo}
 
\begin{resumo}[Abstract]	
\begin{otherlanguage*}{english}
\noindent 
 
This study aims to develop and analyze a realistic simulation of the different phases of sleep in a Spiking Neural Network (SNN)
and their influence on the consolidation and retention of memories. The objective is to better understand the nature of learning
and memory processes in the human brain, and how these processes are affected by the different phases of sleep. To this end, the
characteristics and properties of the different phases of sleep will be investigated to provide a solid foundation for the
simulation. Then, an appropriate SNN model will be selected and used to simulate neural activity during sleep. Finally, the
results of the simulation will be analyzed to determine how consolidation and memory retention are affected by the different
phases of sleep and by neural activity during these phases. It is hoped that this research can contribute to the understanding of
the mechanisms underlying the processes of learning and memory in the human brain and may have significant implications for the
fields of computational neuroscience and artificial intelligence.
 
\vspace{0.2cm}
Keywords: Spiking Neural Networks. Sleep. Memory. Plasticity. Computational Neuroscience.
\end{otherlanguage*}
\end{resumo}
