% resumo em português
\begin{resumo}
\noindent
 
Na interseção entre a neurociência computacional e a Inteligência Artificial (IA), a modelagem de redes neurais oferece uma
perspectiva diferente para o entendimento do cérebro humano. As Redes Neurais Pulsadas (RNP), destacam-se por sua capacidade de
emular mais fielmente a dinâmica dos neurônios biológicos que as tradicionais Redes Neurais Artificiais (RNA). Este estudo se
concentra em um dos desafios ainda não totalmente resolvidos na neurociência: o papel do sono na consolidação e retenção de
memórias. Embora o foco principal seja o avanço do conhecimento neurocientífico, as descobertas também podem ser úteis para certas
áreas do desenvolvimento de sistemas de IA, particularmente aqueles que buscam emular processos cognitivos humanos. A pesquisa
propõe uma investigação sobre como a simulação de fases do sono em RNPs pode influenciar a estabilidade de assembleias neuronais,
unidades fundamentais para o processo de memória. A escolha deste foco é motivada pelo potencial de tais simulações em trazer
novas compreensões sobre a neurociência do sono e suas implicações na aprendizagem e memória. O projeto envolve o desenvolvimento
de um modelo de RNP que incorpora simulações das fases de sono REM e NREM, com base em conhecimentos atuais sobre a neurobiologia
do sono, em uma tarefa de memorização de estímulos visuais. A hipótese é que a simulação dessas fases poderia melhorar a
capacidade da rede de reter e consolidar memórias, comparando-se o desempenho da rede com e sem a simulação do sono. Através da
análise dos efeitos da simulação do sono em RNPs, esperava-se obter uma compreensão mais profunda sobre como as diferentes fases
do sono e a atividade neural durante essas fases afetam a consolidação de memórias. Apesar das expectativas iniciais, os
resultados indicaram que a simulação do sono não produziu mudanças significativas na estabilidade das assembleias neuronais,
sugerindo que o método utilizado não capturou efetivamente a essência dos benefícios do sono observados em organismos biológicos.
Curiosamente, porém, durante as simulações de sono, foi observada uma atividade neuronal que se assemelha a uma espécie de
``sonho'', que, embora tenha sido uma observação interessante, é importante abordá-la com cautela, pois mais pesquisas são
necessárias para compreender esse fenômeno e sua relação com os processos de consolidação da memória durante o sono. Esses achados
apontam para a complexidade do sono e sua modelagem em sistemas computacionais. Eles sugerem que a simulação do sono em RNPs,
embora promissora, ainda requer refinamento e uma melhor compreensão dos processos biológicos subjacentes. 


\vspace{0.2cm}   
Palavras-chave: Redes Neurais Pulsadas. Sono. Memória. Plasticidade. Neurociência Computacional. 
\end{resumo}
 
\begin{resumo}[Abstract]	
\begin{otherlanguage*}{english}
\noindent 
 
In the intersection between computational neuroscience and Artificial Intelligence (AI), neural network modeling offers a
different perspective for understanding the human brain. Spiking Neural Networks (SNNs) stand out for their ability to more
faithfully emulate the dynamics of biological neurons than traditional Artificial Neural Networks (ANNs). This study focuses on
one of the still unresolved challenges in neuroscience: the role of sleep in the consolidation and retention of memories. Although
the main focus is the advancement of neuroscientific knowledge, the findings may also be useful for certain areas of AI system
development, particularly those that seek to emulate human cognitive processes. The research proposes an investigation into how
the simulation of REM and NREM sleep phases in SNNs can influence the stability of cell assemblies, fundamental units for the
memory process. The choice of this focus is motivated by the potential of such simulations to bring new understandings about the
neuroscience of sleep and its implications for learning and memory. The project involves developing an SNN model that incorporates
simulations of REM and NREM sleep phases, based on current knowledge of sleep neurobiology, in a task of memorizing visual
stimuli. The hypothesis is that simulating these phases could improve the network's ability to retain and consolidate memories,
comparing the performance of the network with and without the sleep simulation. Through analyzing the effects of sleep simulation
in SNNs, it was hoped to gain a deeper understanding of how the different phases of sleep and neural activity during these phases
affect memory consolidation. Despite initial expectations, the results indicated that the sleep simulation did not produce
significant changes in the stability of cell assemblies, suggesting that the method used did not effectively capture the
essence of the benefits of sleep observed in biological organisms. Interestingly, however, during the sleep simulations, neuronal
activity resembling a kind of ``dream'' was observed, which, although an interesting observation, is important to approach with
caution, as more research is needed to understand this phenomenon and its relationship with memory consolidation processes during
sleep. These findings point to the complexity of sleep and its modeling in computational systems. They suggest that sleep
simulation in SNNs, while promising, still requires refinement and a better understanding of the underlying biological processes.

\vspace{0.2cm}
Keywords: Spiking Neural Networks. Sleep. Memory. Plasticity. Computational Neuroscience.
\end{otherlanguage*}
\end{resumo}
