Este trabalho visa desenvolver e analisar uma simulação realista das diferentes fases do sono em uma Rede Neural de Disparos (RND)
e sua influência na consolidação e retenção de memórias. Busca-se entender melhor a natureza dos processos de aprendizado e
memória no cérebro humano, e como esses processos são afetados pelas diferentes fases do sono. Para isso, as características e
propriedades das diferentes fases do sono serão investigadas para fornecer uma base sólida para a simulação. Em seguida, um modelo
apropriado de RND será selecionado e utilizado para simular a atividade neural durante o sono. Finalmente, os resultados da
simulação serão analisados para determinar como a consolidação e retenção de memórias são afetadas pelas diferentes fases do sono
e pela atividade neural durante essas fases. Espera-se que essa pesquisa possa contribuir para a compreensão dos mecanismos
subjacentes aos processos de aprendizado e memória no cérebro humano e possa ter implicações significativas para os campos da
neurociência computacional e inteligência artificial.

% Separe as palavras-chave por ponto
\palavraschave{Redes Neurais de Disparos. Sono. Memória. Plasticidade. Neurociência Computacional.}