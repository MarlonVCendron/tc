\chapter{Trabalhos correlatos}


% @Plasticidade As principais limitações encontradas nestes trabalhos estão relacionadas com ….


% @Sono bla bla bla bla. A principal vantagem é que o sono ajuda a melhorar a memória


A utilização de modelos computacionais de neurônios e RNDs para o estudo das funções cognitivas do cérebro é muito popular, pois
em simulações computacionais existe o controle total sobre todas as propriedades de cada neurônio simulado


No trabalho de~\cite{}, os autores criaram uma RND composta de 4096 neurônios excitatórios e 1024 neurônios inibitórios para
testar a hipótese de que diferentes formas de plasticidade, tanto hebbianas como não hebbianas, quando implementadas juntas
poderiam levar à formação e à recordação de conjuntos celulares. O modelo criado pôde criar memórias dos quatro diferentes
estímulos visuais que recebia, e essas memórias eram estáveis e podiam ser recordadas até mesmo depois de horas de simulação sem
que o modelo visse novamente os estímulos. Esse modelo de RND serviu para provar que é possível, pelo menos em teoria, ter a
formação de conjuntos celulares, e consequentemente memória, apenas a partir de mecanismos de plasticidade orquestrados. Porém,
algo que os autores não exploraram, ou pelo menos não tornaram explícito em seu trabalho, é a quantidade de estímulos que a RND é
capaz de memorizar; também no mesmo tema, não é explorado o que acontece com o modelo quando um novo estímulo é apresentado ao
depois da memorização dos demais.

No trabalho não lembro qual, os autores utilizam RNDs para um problema de não lembro o que. Ao tentar treinar a RND em duas
tarefas diferentes, a rede experienciava um fenômeno chamado de "Catastrophic Forgetting", em que uma das tarefas aprendidas era
completamente esquecida após aprender a outra. Ao implementar um estado que os autores chamaram de sono na rede, caracterizado por
curtos períodos de não lembro o que, a rede se tornou capaz de aprender ambas as tarefas sem esquecer do que já havia aprendido
anteriormente. Essa simulação


% Talvez referenciar goldenSleep2022?  {Sleep Prevents Catastrophic Forgetting in Spiking Neural Networks by Forming a Joint Synaptic Weight Representation},

