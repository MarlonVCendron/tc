\section{Redes Neurais de Disparo}

Juntando os modelos de neurônios e sinapses descritos nas seções~\ref{section_modelos_neuronios}
e~\ref{subsection_modelos_sinapses}, podemos construir modelos de redes neurais de disparo (RNDs). As RNDs são modelos
computacionais que buscam emular a forma como os neurônios biológicos interagem e se comunicam entre si no
cérebro~\cite{maassNetworks1997}. Ao contrário dos modelos de redes neurais artificiais tradicionais, como as redes
\emph{feed-forward} e as redes neurais convolucionais, que lidam com dados de entrada e saída de maneira contínua e uniforme, as
RNDs introduzem um elemento de tempo e uma representação mais precisa do disparo dos neurônios.




