\section{Redes Neurais de Disparo}

Juntando os modelos de neurônios e sinapses descritos nas seções~\ref{section_modelos_neuronios}
e~\ref{subsection_modelos_sinapses}, podemos construir modelos de redes neurais de disparo (RNDs). As RNDs são modelos
computacionais que buscam emular a forma como os neurônios biológicos interagem e se comunicam entre si no
cérebro~\cite{maassNetworks1997}. Ao contrário dos modelos de redes neurais artificiais tradicionais, como as redes
\emph{feed-forward} e as redes neurais convolucionais, que lidam com dados de entrada e saída de maneira contínua e uniforme, as
RNDs introduzem um elemento de tempo e uma representação mais precisa do disparo dos neurônios.

% Redes neurais de disparo são modelos computacionais que buscam emular a forma como os neurônios biológicos interagem e se
% comunicam entre si no cérebro~\cite{maassNetworks1997}. Ao contrário dos modelos de redes neurais artificiais tradicionais, como
% as redes feed-forward e as redes neurais convolucionais, que lidam com dados de entrada e saída de maneira contínua e uniforme, as
% RNDs introduzem um elemento de tempo e uma representação mais precisa do disparo dos neurônios ao fazer uso de modelos de
% neurônios descritos na seção~\ref{section_modelos_neuronios}.

% Nas RNDs, cada neurônio recebe sinais de entrada de outros neurônios na forma de pulsos, ou disparos, que são transmitidos ao
% longo de conexões chamadas sinapses~\cite{gerstnerSpikingNeuronModels2002}. A intensidade do sinal transmitido através de uma
% sinapse pode variar, o que é conhecido como peso sináptico. Este peso pode ser ajustado ao longo do tempo, em um processo que se
% assemelha ao aprendizado por meio da plasticidade sináptica no cérebro biológico.

% O neurônio acumula os pulsos de entrada ao longo do tempo em uma soma ponderada, em um processo que se assemelha à integração de
% sinais de entrada em um neurônio biológico. Quando a soma de entrada de um neurônio ultrapassa um certo limiar, o neurônio
% dispara, enviando um pulso de saída a todos os neurônios a que está conectado. A introdução deste mecanismo de disparo e de tempo
% nas redes neurais permite que as RNDs representem e processem informações de maneira mais dinâmica e realista.

% O modo como os neurônios se conectam nas RNDs é uma questão crucial para a estrutura e a funcionalidade da rede. Em um cérebro
% biológico, cada neurônio está conectado a muitos outros, formando uma rede densa e complexa. Nas RNDs, é comum adotar uma
% estrutura de conexão em camadas, em que os neurônios são organizados em grupos, ou camadas, e os neurônios de uma camada se
% conectam aos neurônios da camada seguinte. Esta estrutura em camadas pode facilitar a aprendizagem e a generalização em tarefas
% complexas.

% No entanto, também é possível adotar estruturas de conexão mais complexas e realistas, como redes com conexões recorrentes, onde
% os neurônios podem se conectar a outros dentro da mesma camada ou até mesmo a si mesmos. Estas redes podem capturar dinâmicas
% temporais mais complexas e são utilizadas para modelar aspectos mais sofisticados do processamento de informação no
% cérebro~\cite{izhikevichPolychronization2006}.





