\section{Conjuntos Celulares}

A plasticidade dá origem a um fenômeno emergente no cérebro chamado de conjuntos celulares, fenômeno em que grupos de neurônios relacionados
a um mesmo estímulo ou processo acabam fortalecendo as conexões entre si e que podem servir diversas funções, como pequenas unidades de
processamento, memória associativa, entre outras.

Tomando um exemplo simplificado da memória de uma viagem à praia: essa memória consiste em vários elementos, como o som das ondas,
a sensação de areia quente, o cheiro de água salgada e a visão de gaivotas etc. Cada um desses elementos sensoriais é processado em diferentes
áreas do cérebro e ativa diferentes grupos  de neurônios. No momento da formação da memória, os neurônios ou grupos de neurônios responsáveis
por esses elementos sensoriais disparam ao mesmo tempo, criando uma correlação entre eles. Essa correlação é o que desencadeia a plasticidade,
esses neurônios terão então as conexões entre si fortalecidas. Com a memória formada, em um momento futuro em que o indivíduo com a memória ouça
novamente o som das ondas, por exemplo, por conta da agora forte conexão dos neurônios do estímulo sonoro das ondas com as demais características
da memória codificada no conjunto celular, é possível que os neurônios relacionados com a sensação da areia, com o cheiro da água etc. também sejam
ativados, resultando então na experiência da memória.

O exemplo dado no parágrafo anterior é bem simplificado, servindo apenas para entender como funciona a formação de conjuntos celulares e a sua relação
com as memórias. O cérebro humano é muito mais complexo e possui muito mais neurônios, não necessariamente vai haver uma conexão direta entre um
neurônio que ativa para o conceito de ondas e um neurônio que ativa para o conceito de areia, por exemplo, muitas vezes nem existe um neurônio único
ou um grupo de neurônios específicos que delimitam o conceito no cérebro; também, a formação de memórias no cérebro não ocorre apenas pelo simples
funcionamento da plasticidade, embora esse seja o mecanismo por trás de tudo, no cérebro existem órgãos (????? outra palavra no lugar de órgão)
específicos que mediam a formação de memórias, como o hipocampo por exemplo, que possui como uma de suas funções conhecidas a de repetir diversas
vezes estímulos no córtex de modo a fixar memórias de longo prazo\cite{}.

% <imagem demonstrando que diferentes conjuntos celulares representam memórias diferentes>
