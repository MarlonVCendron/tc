\section{Conjuntos Celulares}

A plasticidade hebbiana pode ser sumarizada pela seguinte frase "Neurônios que disparam juntos, conectam-se juntos", assim, a
plasticidade dá origem a um fenômeno emergente no cérebro chamado de conjuntos celulares\footnote{Do inglês \textit{Cell
Assemblies}. Também traduzido como Assembleias Celulares.}, em que grupos de neurônios relacionados a um mesmo estímulo ou
processo acabam fortalecendo as conexões entre si. Um conjunto celular pode servir como uma forma de memória associativa
\cite{sakuraiMultiple2018}, em que a ativação de um dos neurônios do conjunto acaba por ativar sincronamemente os demais neurônios
do conjunto devido ao fortalecimento das conexões entre os neurônios do conjunto. Tomando como exemplo a memória de uma viagem à
praia: essa memória consiste em vários elementos, como o som das ondas, a sensação de areia quente, o cheiro de água salgada e a
visão de gaivotas etc. Cada um desses elementos sensoriais é processado em diferentes áreas do cérebro e ativa diferentes grupos
de neurônios. A ativação síncrona dos neurônios responsáveis por esses elementos sensoriais leva à formação de um conjunto
celular. Um tempo depois, ao sentir o cheiro do mar novamente, esse estímulo pode acabar ativando o conjunto celular, resultando
na experiência da memória.

Um conjunto celular não é uma estrutura estática, mas sim uma rede dinâmica de neurônios que podem ser modificados e reativados ao
longo do tempo. A falta de estímulos ou a exposição a novas informações pode levar a uma modificação ou mesmo ao esquecimento de
partes da memória {...} 

<imagem demonstrando que diferentes conjuntos celulares representam memórias diferentes>