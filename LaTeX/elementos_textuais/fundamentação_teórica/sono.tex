\section{Sono}

O sono é um processo fisiológico crucial para a consolidação e manutenção das memórias~\cite{blissittSleep2001, walkerSleep2006, diekelmannMemory2010}.
Inicialmente, postulava-se que o sono desempenhava uma
função passiva no processo de consolidação da memória~\cite{jenkinsObliviscence1924}; contudo, com a descoberta das distintas
fases do sono, começaram-se a explorar as contribuições ativas do sono na consolidação mnemônica~\cite{aserinskyRegularly1953}.
Durante o sono, ocorrem diferentes fases caracterizadas por padrões distintos de atividade cerebral: sono REM (\textit{Rapid Eye
Movement}) e sono não REM (NREM, dividido entre as fases N1, N2 e N3)~\cite{schulzRethinking2008}. Durante a fase NREM, oscilações
lentas, fusos e ondulações coordenam a reativação e redistribuição de memórias dependentes do hipocampo para o
neocórtex~\cite{diekelmannMemory2010}. Já quanto ao sono REM, a dificuldade em isolar a atividade neural dessa etapa específica,
que ocorre após a fase NREM, torna a discussão sobre sua contribuição para a consolidação da memória ainda controversa. Contudo,
pesquisas mais recentes oferecem evidências de que o sono REM desempenha um papel fundamental na consolidação da memória espacial
e contextual~\cite{boyceREM2017}.