\section{Neurônios biológicos}

Neurônios recebem sinais dos dendritos, esses sinais são integrados no soma e, se for o caso, um potencial de ação é disparado
pelo axônio. O axônio carrega o sinal que é transmitido para os dendritos de outros neurônios através da sinapse.

Sinapses podem ser químicas (através de neurotransmissores), mais lentas mas que podem intesificar o sinal, ou elétricas, rápidas
mas que não podem mudar a amplitude do sinal. Sinapses podem ser excitatórias, que despolarizam a célula pós-sináptica, ou
inibitórias, que hiperpolarizam. Os neurotransmissores excitatórios mais comuns são: glutamato, dopamina e noradrenalina. Os
inibitórios são: GABA, glicina e serotonina.

O potencial de membrana é o potencial elétrico interior da célula em relação ao exterior. Em neurônios, o potencial de repouso é
de -70,15 mV. Quando ocorre um potencial de ação, canais de Na+ são abertos até que o potencial de reversão do sódio é alcançado,
em 38,43 mV. O potencial de membrana cai até chegar no potencial de reversão do K+ e hiperpolariza um pouco além do potencial de
repouso, para então voltar ao potencial de repouso. Diferentes neurônios têm potencial de repouso e de reversão de sódio e
potássio diferentes, entre mais ou menos 10 mV.

// mais detalhes de neurônios
{...}