\section{Neurônios biológicos}\label{sec_neurônios}

Neurônios são células especializadas que atuam como unidades básicas de comunicação no cérebro, transmitindo informações através
de impulsos elétricos e conexões químicas e elétricas, chamadas sinapses. Um neurônio possui três principais partes, como ilustra
a figura~\ref{fig_neuron}: dendritos, soma e axônio. Dendritos recebem sinais de outros neurônios e os conduzem ao soma,
onde são integrados. Se o sinal integrado atinge um limiar, um potencial de ação, ou disparo, é gerado no axônio, propagando-se
até as terminações axonais através das sinapses, onde o sinal se propaga para os dendritos dos neurônios seguintes. A
figura~\ref{fig_action_potential} mostra a curva característica de um potencial de ação.

\begin{figure}[!ht]
\Caption{\label{fig_neuron}Neurônio piramidal neocortical humano. O axônio é claramente vísivel se extendendo verticalmente a partir do soma, assim
como os vários dendritos que partem da base do soma.}
\centering
\includegraphics[width=12cm]{figuras/neuron.png}
\Fonte{\cite{jacobsSeeing}}
\end{figure}

O potencial de membrana é o potencial elétrico interior da célula em relação ao exterior. Em repouso, um neurônio possui um
potencial de membrana negativo, de -40mV a -80mV, chamado potencial de repouso, que é mantido através da bomba de sódio-potássio e
da permeabilidade seletiva da membrana. Em vários neurônios, uma despolarização de aproximadamente 10mV é o suficiente para
atingir o limiar de excitação, que, uma vez atingido, desencadeia a abertura dos canais de sódio (Na+) voltagem-dependentes,
permitindo a entrada de mais Na+ e causando uma maior despolarização. Quando o potencial de membrana atinge seu pico, normalmente
por volta de +40mV, os canais de Na+ se fecham, enquanto os canais de potássio (K+) voltagem-dependentes se abrem, permitindo a
saída de K+ e repolarizando a membrana. Esse processo é seguido por uma hiperpolarização temporária, antes que o potencial de
membrana retorne ao seu estado de repouso. Essa sequência de eventos constitui um potencial de ação, que se propaga ao longo do
axônio até as sinapses, permitindo a comunicação entre os neurônios~\cite{kandelPrinciples2021}.

\begin{figure}[!ht]
\Caption{\label{fig_action_potential}Potencial de ação, linha vermelha, condutância da membrana, envelope.}
\centering
\includegraphics[width=8.5cm]{figuras/Action Potential.png}
\Fonte{\cite{coleELECTRIC1939}. Modificada pelo autor.}
\end{figure}

As sinapses podem existir de duas maneiras distintas: por meio de transmissão química, utilizando neurotransmissores, ou por
transmissão elétrica. Embora a transmissão química seja mais lenta, ela pode intensificar o sinal transmitido, enquanto a
transmissão elétrica é mais rápida, mas não é capaz de modificar a amplitude do sinal. Esse trabalho irá focar apenas nas sinapses químicas, 
já que essas são muito mais abundantes no cérebro. As sinapses podem ter um efeito
excitatório, resultando em despolarização da célula pós-sináptica, ou um efeito inibitório, resultando em hiperpolarização. Entre
os neurotransmissores mais comuns que causam efeito excitatório estão o glutamato, a dopamina e a noradrenalina, enquanto o GABA,
a glicina e a serotonina são exemplos de neurotransmissores que exercem efeito inibitório.

Quando um potencial de ação chega à terminação axonal do neurônio pré-sináptico, vesículas contendo neurotransmissores são
liberadas na fenda sináptica. Os neurotransmissores se ligam a receptores específicos na membrana do neurônio pós-sináptico,
ativando ou inibindo os canais iônicos. Se o neurotransmissor for excitatório, ele induz a abertura de canais iônicos como os de
Na+ e Ca2+, resultando em uma entrada líquida de íons positivos e uma despolarização da membrana pós-sináptica. Por outro lado, se
o neurotransmissor for inibitório, ele geralmente causa a abertura de canais de K+ e/ou Cl-, levando à saída de íons K+ ou entrada
de íons Cl-, o que resulta em uma hiperpolarização da membrana pós-sináptica.