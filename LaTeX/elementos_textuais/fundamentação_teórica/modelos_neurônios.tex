\section{Modelos de Neurônios}

!!! Na verdade Lapicque já tinha criado um modelo matemático em 1907. Modelo IF

Lapicque L (1907) Quantitative investigations of electrical
nerve excitation treated as polarization (in French). J
Physiol Pathol Gen 9: 620–635.

O modelo de neurônio de McCulloh-Pitts \cite{mccullochLogical1943} representou a primeira tentativa de representar matematicamente
um neurônio biológico. O modelo é bastante simplificado: o neurônio recebe X sinais de entrada binários, que podem ser
excitatórios (+1) ou inibitórios (-1). Esses sinais são integrados e, em seguida, é verificado se um limiar é atingido. Apenas
quando o limiar é ultrapassado, o neurônio emite um sinal. O modelo foi uma ideia inovadora ao tentar analisar a biologia por meio
de termos matemáticos. No entanto, o modelo não conseguiu capturar toda a complexidade de um neurônio biológico, pois essa não era
a intenção do trabalho. O objetivo principal era mostrar que era possível representar e analisar o sistema nervoso usando lógica
proposicional.

<equação>
<imagem mcculloh-pitts neuron>

O modelo de Hodgkin-Huxley \cite{hodgkinQuantitative1952}, representou um avanço significativo na compreensão e na modelagem dos
neurônios biológicos. Esse modelo buscou descrever a geração e propagação de potenciais de ação em neurônios, levando em
consideração os processos eletroquímicos subjacentes, como a dinâmica dos canais iônicos e as correntes iônicas que atravessam a
membrana celular. Diferentemente do modelo de McCulloch-Pitts, o modelo de Hodgkin-Huxley é capaz de descrever disparos de
potenciais de ação, fornecendo uma representação mais precisa e detalhada da atividade neuronal.

O modelo de Hodgkin-Huxley é composto por um conjunto de equações diferenciais ordinárias que descrevem a variação do potencial de
membrana em função do tempo e das correntes iônicas. Essas equações consideram o comportamento dinâmico dos canais iônicos de
sódio e potássio, bem como a corrente de vazamento através da membrana. O modelo é capaz de capturar o comportamento típico dos
neurônios, incluindo a resposta ao estímulo, a fase refratária e a propagação do sinal ao longo do axônio.

<equações>
<imagem comparando tensão de neurônio biológico e de Hodgkin-Huxley>

citar Stein 1967

Os modelos de neurônios do tipo Integrate-and-Fire (IF), como o modelo Leaky Integrate-and-Fire (LIF) \cite{burkitt2006review},
surgiram como uma alternativa mais simples e computacionalmente eficiente em comparação ao modelo de Hodgkin-Huxley. Embora não
sejam tão biologicamente precisos quanto o modelo de Hodgkin-Huxley, os modelos IF conseguem capturar algumas das características
essenciais dos neurônios, como a integração temporal dos estímulos e a emissão de potenciais de ação quando um limiar é atingido.
Outra vantagem dos modelos IF é que modelos simples são uma forma de reduzir a complexidade do cérebro para seus mecanismos mais
fundamentais.

<equação>