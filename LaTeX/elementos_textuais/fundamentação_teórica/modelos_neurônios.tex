\section{Modelos de Neurônios}\label{section_modelos_neuronios} 

Em 1907, Lapicque desenvolveu um modelo de neurônio que descreve o neurônio como um circuito elétrico contendo um capacitor e um
resistor em paralelo~\ref{imagem}, representando a capacitância e a resistência de vazamento da membrana
celular~\cite{lapicqueRecherches1907}. Mesmo sem entender os mecanismos por trás da geração de potenciais de ação, Lapicque
postulou que, ao atingir um certo potencial limiar, um potencial de ação seria gerado e o capacitor descarregado, reiniciando o
potencial da membrana. Isso mostra que, ao se tratar de modelagem de neurônios, estudos da função não necessariamente requerem
conhecimento do mecanismo~\cite{abbottLapicque1999}. O modelo de neurônio de Lapicque foi a primeira tentativa de representar
matematicamente um neurônio biológico.

<imagem circuito>

O modelo de Hodgkin-Huxley~\cite{hodgkinQuantitative1952}, representou um avanço significativo na compreensão e na modelagem dos
neurônios biológicos. Diferentemente do primeiro modelo criado por Lapicque, esse modelo buscou descrever a geração e propagação
de potenciais de ação em neurônios levando em consideração os processos eletroquímicos subjacentes, como a dinâmica dos diferentes
canais iônicos que controlam a corrente elétrica através da membrana celular.

O modelo de Hodgkin-Huxley é composto por um conjunto de equações diferenciais ordinárias que descrevem a variação do potencial de
membrana em função do tempo e das correntes iônicas. Essas equações consideram o comportamento dinâmico dos canais iônicos de
sódio e potássio, bem como a corrente de vazamento através da membrana. O modelo é capaz de capturar o comportamento típico dos
neurônios, incluindo a resposta ao estímulo, a fase refratária e a propagação do sinal ao longo do axônio.

<equações>

<imagem comparando tensão de neurônio biológico e de Hodgkin-Huxley>

Baseados na ideia de Lapicque, hoje em dia são utilizados os modelos Integrate-and-Fire (IF). Existem diversos modelos IF, com
várias modificações da ideia original, como o modelo Leaky Integrate-and-Fire (LIF)~\cite{burkitt2006review}. Os modelos IF são
uma alternativa mais simples e computacionalmente eficiente em comparação ao modelo de Hodgkin-Huxley. Embora não sejam tão
biologicamente precisos quanto o modelo de Hodgkin-Huxley, os modelos IF conseguem capturar algumas das características essenciais
dos neurônios, como a integração temporal dos estímulos e a emissão de potenciais de ação quando um limiar é atingido. Outra
vantagem dos modelos IF é que modelos simples são uma forma de reduzir a complexidade do cérebro para seus mecanismos mais
fundamentais.

% Explicar mais detalhadamente o modelo LIF

<equação>

Existem diversos outros modelos de neurônios, como o modelo de Izhikevich~\cite{izhikevichSimple2003}, que é quase tão simples em
termos computacionais quanto o modelo IF, mas que consegue capturar de forma muito mais realista um conjunto maior de
comportamentos neurais dependendo dos parâmetros utilizados. Esse modelo é especialmente útil para simular neurônios específicos
do cérebro, enquanto os modelos IF são preferidos quando o objetivo é estudar o comportamento geral de neurônios por sua
simplicidade, e, portanto, serão utilizados nesse trabalho.
