\section{Modelos de Neurônios}\label{section_modelos_neuronios} 

Em 1907, Lapicque desenvolveu um modelo de neurônio que descreve o neurônio como um circuito elétrico contendo um capacitor e um
resistor em paralelo, como representado na figura~\ref{fig_lapicque}, representando a capacitância e a resistência de vazamento da
membrana celular~\cite{lapicqueRecherches1907}, chamado de modelo Integrate-and-Fire (IF). Mesmo sem entender os mecanismos por
trás da geração de potenciais de ação, Lapicque postulou que, ao atingir um certo potencial limiar, um potencial de ação seria
gerado e o capacitor descarregado, reiniciando o potencial da membrana. Isso mostra que, ao se tratar de modelagem de neurônios,
estudos da função não necessariamente requerem conhecimento do mecanismo~\cite{abbottLapicque1999}. O modelo de neurônio de
Lapicque foi a primeira tentativa de representar matematicamente um neurônio biológico.

\begin{figure}[!ht]
\Caption{\label{fig_lapicque} (A) O circuito elétrico de Lapicque: I é a corrente injetada, C a capacitância da membrana, R a resistência da membrana,
V o potencial de membrana e $V_{rest}$ o potencial de repouso. (B) A trajetória de tensão, quando um limiar é atingido, um
potencial de ação é disparado. (C) Um modelo IF com corrente que varia pelo tempo.}
\centering
\includegraphics[width=\linewidth]{figuras/lapicque.png}
\Fonte{\cite{abbottLapicque1999}}
\end{figure}

O modelo de Hodgkin-Huxley~\cite{hodgkinQuantitative1952}, representou um avanço significativo na compreensão e na modelagem dos
neurônios biológicos. Diferentemente do primeiro modelo criado por Lapicque, esse modelo buscou descrever a geração e propagação
de potenciais de ação em neurônios levando em consideração os processos eletroquímicos subjacentes, como a dinâmica dos diferentes
canais iônicos que controlam a corrente elétrica através da membrana celular.

O modelo de Hodgkin-Huxley é composto por um conjunto de equações diferenciais ordinárias que descrevem a variação do potencial de
membrana em função do tempo e das correntes iônicas. Essas equações consideram o comportamento dinâmico dos canais iônicos de
sódio e potássio, bem como a corrente de vazamento através da membrana. O modelo é capaz de capturar o comportamento típico dos
neurônios, incluindo a resposta ao estímulo, a fase refratária e a propagação do sinal ao longo do axônio.

Baseados na ideia de Lapicque, hoje em dia são utilizados os modelos IF.\ Existem diversos modelos IF, com
várias modificações da ideia original, como o modelo Leaky Integrate-and-Fire (LIF)~\cite{burkittReview2006}. Os modelos IF são
uma alternativa mais simples e computacionalmente eficiente em comparação ao modelo de Hodgkin-Huxley. Embora não sejam tão
biologicamente precisos quanto o modelo de Hodgkin-Huxley, os modelos IF conseguem capturar algumas das características essenciais
dos neurônios, como a integração temporal dos estímulos e a emissão de potenciais de ação quando um limiar é atingido. Outra
vantagem dos modelos IF é que modelos simples são uma forma de reduzir a complexidade do cérebro para seus mecanismos mais
fundamentais.

Apesar de sua simplicidade, os modelos IF têm sido amplamente utilizados em RND devido à sua eficiência computacional e capacidade
de reproduzir aspectos fundamentais do comportamento neuronal. Por exemplo, no trabalho de~\cite{teeterGeneralized2018} o
comportamento de 645 neurônios do neocórtex foi reproduzido utilizando modelos GLIF (Generalized Leaky Integrate-and-Fire). 

O modelo de neurônio LIF é descrito pela dinâmica do potencial de membrana do neurônio, $v(t)$, que é dado pela
equação~\ref{eq_lif}

\begin{equation}
\label{eq_lif}
C_m \frac{dv(t)}{dt} = -\frac{C_m}{\tau_m} [v(t) - V_0] + I(t)
\end{equation}

onde $C_m$ é a capacitância da membrana, $V_0$ é o potencial de repouso, $\tau_m$ é a constante de tempo passiva da membrana
(relacionada à capacitância do neurônio e à resistência de vazamento do potencial de membrana por $\tau_m = R_m C_m$), $I(t)$ é a
corrente elétrica injetada no neurônio (tanto a corrente causada pelas sinapses, como a por eletrodos)~\cite{burkittReview2006}. 

Quando o potencial de membrana atinge um limiar, $V_{th}$, um potencial de ação é disparado e o potencial de membrana é resetado
para $V_{reset}$, o potencial um pouco menor que o potencial de repouso, correspondente ao período refratário do neurônio.

Existem diversos outros modelos de neurônios, como o modelo de Izhikevich~\cite{izhikevichSimple2003}, que é quase tão simples em
termos computacionais quanto o modelo IF, mas que consegue capturar de forma muito mais realista um conjunto maior de
comportamentos neurais dependendo dos parâmetros utilizados. Esse modelo é especialmente útil para simular neurônios específicos
do cérebro, enquanto os modelos IF são preferidos quando o objetivo é estudar o comportamento geral de neurônios por sua
simplicidade, e, portanto, serão utilizados nesse trabalho.

\begin{figure}[!ht]
\Caption{Diferentes tipos de neurônios simulados pelo modelo de Izhikevich.}
\centering\label{fig_izhikevich}
\includegraphics[width=\linewidth]{figuras/izhikevich.png}
\Fonte{\cite{izhikevichSimple2003}}
\end{figure}

\subsection{Modelos de Sinapses}\label{subsection_modelos_sinapses}

% Quando se deseja simular uma rede neural, é necessário considerar não apenas os modelos de neurônios, mas também os modelos de
% sinapses. As sinapses são as conexões entre os neurônios, e são responsáveis pela transmissão de sinais entre eles.

% O modelo de sinapse mais simples, chamado modelo de sinapse de pulso delta, considera a sinapse como um ponto de interação
% instantâneo entre neurônios. Nesse modelo, o potencial pós-sináptico é instantaneamente aumentado por uma quantidade fixa cada vez
% que um pulso pré-sináptico ocorre. Embora esse modelo seja computacionalmente eficiente, ele é uma simplificação extrema da
% complexidade biológica da sinapse~\cite{gerstner2002spiking}.

% Para uma representação mais biologicamente realista, os modelos de sinapses de tempo contínuo podem ser utilizados. Estes
% consideram o retardo sináptico e a duração da resposta pós-sináptica. No modelo de condutância baseado em tempo, a sinapse é
% representada por uma condutância variável que aumenta após a chegada de um pulso pré-sináptico e decai exponencialmente com o
% tempo~\cite{abbott2004chemically}. Este modelo é capaz de representar os efeitos de vários impulsos pré-sinápticos que chegam em
% tempos próximos, um fenômeno conhecido como facilitação sináptica.

% Outro modelo que se destaca é o modelo de sinapse de Markram-Tsodyks~\cite{markram1997synaptic}, que aborda a plasticidade de
% curto prazo da sinapse, um processo pelo qual a eficácia de uma sinapse em transmitir sinais entre neurônios pode aumentar ou
% diminuir em um curto período de tempo. Este modelo é particularmente útil para simular sinapses em redes neuronais que apresentam
% dinâmica complexa, como o córtex cerebral. Apesar da complexidade adicional, esses modelos de sinapses são necessários para
% capturar as sutilezas da interação entre neurônios em uma rede neural.




