\section{Objetivos específicos}

Sugerir uma abordagem inspirada no sono para retenção de memórias em RND

Desenvolver uma simulação de sono inspirada nas observações de ondas cerebrais durante as diferentes fases do sono.

Avaliar a capacidade de memória da rede em relação aos padrões de imagens apresentados à ela.

Comparar a performance da RND com e sem a simulação do sono, a fim de verificar a hipótese de melhoria na capacidade de memória.

Analisar os resultados e discutir possíveis implicações e aplicações nos campos de neurociência computacional e inteligência
artificial.



1 - Metodologia
2 - Depende do problema: levantamento de dados. Provavelmente fique de fora
3 - Condução de experimentos
4 - Avaliação do modelo (protocolo de avaliação)
5 - Comparação com estado da arte
