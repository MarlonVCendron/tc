\section{Experimentos}\label{section_experimento}

Como forma de avaliar o modelo proposto serão conduzidos dois principais experimentos.

\subsection{Experimento 1: Formação de assembleias neuronais}

O primeiro experimento consiste em simular a RNP apresentando estímulos ao modelo de retina e analisar se a repetição dos
estímulos leva à formação de assembleias neuronais associadas a cada estímulo a longo prazo, com o objetivo de ter uma base de
comparação para o experimento 2, que terá o sono simulado. Os estímulos consistem em seis imagens simples de serem reconhecidas,
exibidas na Figura~\ref{fig_estimulos}, e são apresentados à rede de forma intercalada e aleatória. Quatro estímulos foram
reaproveitados do trabalho de~\citeonline{zenkeDiverse2015}, enquanto as figuras de diamante e de cruz foram adicionadas com o
intuito de colocar a RNP mais próxima do seu limite.

\begin{figure}[!ht]
\caption{Os seis estímulos apresentados à RNP durante os experimentos.}
\centering{
\parbox{12cm}{
\includegraphics[width=12cm]{figuras/estimulos_cor.png}\label{fig_estimulos}
\fonte{Elaborado pelo autor (2023).}}}
\end{figure}

Houveram três fases da simulação da RNP:

\begin{enumerate}
  \item Simulação da rede em seu estado inicial por 1800 segundos, com um tempo médio de aparição do estímulo de 2 segundos, e
  tempo médio entre estímulos de 1 segundo. O peso das sinapses entre a retina e a rede é de 0.05.
  \item Simulação da rede também por 1800 segundos, mas com um tempo médio de aparição do estímulo de 0.2 segundo, e
  tempo médio entre estímulos de 5 segundos. O peso das sinapses entre a retina e a rede agora é de 0.1. A intenção aqui é
  fazer com que a rede tenha mais tempo entre um estímulo e outro para poder memorizá-los melhor.
  \item A última simulação é de 2400 segundos e é de onde são tirados os resultados. O tempo médio de aparição do estímulo diminui
  para 0.1, enquanto o tempo médio entre estímulos é de 10 segundos, com a intenção de ter uma janela maior de tempo entre os
  estímulos para analisar a capacidade de memorização da RNP.
\end{enumerate}


\subsection{Experimento 2: Formação de assembleias neuronais com sono}

De forma similar ao experimento 1, o segundo experimento consiste em simular a RNP apresentando os mesmos estímulos, mas dessa vez
com a simulação de sono. O objetivo desse experimento é analisar se a simulação de sono tem algum efeito na formação de
assembleias neuronais.

Esse experimento seguiu as mesmas três fases do experimento anterior, mas com a simulação do sono. Nesse experimento, a RNP ficava
em estado de vigília por 400s e dormia por 200s.


\section{Análise dos resultados}

O número médio de neurônios por assembleia neuronal na RNP base foi de 709, enquanto na RNP com sono foi 573.5; isso se deve
principalmente ao fato de que a assembleia neuronal do círculo teve pouquíssimos neurônios, apenas 109 na RNP com sono. Ambas as
RNPs tiveram dificuldades em manter uma assembleia neuronal única para o estímulo do círculo, havendo bastante sobreposição com os
neurônios da assembleia neuronal do quadrado; isso provavelmente ocorre pela similaridade entre os dois estímulos e também a
possibilidade da RNP ter alcançado um limite de memória. Também ocorre um pouco dessa sobreposição na RNP com sono entre a
assembleia neuronal do quadrado e do triângulo. A matriz de sobreposição das assembleias neuronais está ilustrada nas
Figura~\ref{fig_mat_base}.

\begin{figure}[!ht]
\caption{Matrizes contendo o número de neurônios pertecentes a cada assembleia neuronal. Os elementos da diagonal principal da
matriz indicam o número total de neurônios em cada assembleia individual. Por outro lado, o elemento na posição $i,j$ da matriz
mostra quantos neurônios da assembleia $i$ também são parte da assembleia $j$.}

\centering{
\parbox{\linewidth}{
\includegraphics[width=7.5cm]{figuras/plots/mat_base.png}\label{fig_mat_base}
\hfill
\includegraphics[width=7.5cm]{figuras/plots/mat_sleep.png}\label{fig_mat_sleep}
\fonte{Elaborado pelo autor (2023).}}}
\end{figure}

