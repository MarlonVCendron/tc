\section{Experimentos}\label{section_experimento}

Como forma de avaliar o modelo proposto serão conduzidos dois principais experimentos.

\subsection{Experimento 1: Formação de assembleias neuronais}

O primeiro experimento consiste em simular a RNP apresentando estímulos ao modelo de retina e analisar se a repetição dos
estímulos leva à formação de assembleias neuronais associadas a cada estímulo a longo prazo, com o objetivo de ter uma base de
comparação para o experimento 2, que terá o sono simulado. Os estímulos consistem em seis imagens simples de serem reconhecidas,
exibidas na Figura~\ref{fig_estimulos}, e são apresentados à rede de forma intercalada e aleatória. Quatro estímulos foram
reaproveitados do trabalho de~\citeonline{zenkeDiverse2015}, enquanto as figuras de diamante e de cruz foram adicionadas com o
intuito de colocar a RNP mais próxima do seu limite.

\begin{figure}[!ht]
\caption{Os seis estímulos apresentados à RNP durante os experimentos.}
\centering{
\parbox{12cm}{
\includegraphics[width=12cm]{figuras/estimulos_cor.png}\label{fig_estimulos}
\fonte{Elaborado pelo autor (2023).}}}
\end{figure}

Houveram três fases da simulação da RNP:

\begin{enumerate}
  \item Simulação da rede em seu estado inicial por 1800 segundos, com um tempo médio de aparição do estímulo de 2 segundos, e
  tempo médio entre estímulos de 1 segundo. O peso das sinapses entre a retina e a rede é de 0.05.
  \item Simulação da rede também por 1800 segundos, mas com um tempo médio de aparição do estímulo de 0.2 segundo, e
  tempo médio entre estímulos de 5 segundos. O peso das sinapses entre a retina e a rede agora é de 0.1. A intenção aqui é
  fazer com que a rede tenha mais tempo entre um estímulo e outro para poder memorizá-los melhor.
  \item A última simulação é de 2400 segundos e é de onde são tirados os resultados. O tempo médio de aparição do estímulo diminui
  para 0.1, enquanto o tempo médio entre estímulos é de 10 segundos, com a intenção de ter uma janela maior de tempo entre os
  estímulos para analisar a capacidade de memorização da RNP.
\end{enumerate}


\subsection{Experimento 2: Formação de assembleias neuronais com sono}

De forma similar ao experimento 1, o segundo experimento consiste em simular a RNP apresentando os mesmos estímulos, mas dessa vez
com a simulação de sono. O objetivo desse experimento é analisar se a simulação de sono tem algum efeito na formação de
assembleias neuronais.

Esse experimento seguiu as mesmas três fases do experimento anterior, mas com a simulação do sono. Nesse experimento, a RNP ficava
em estado de vigília por 200s e dormia por 100s, seguindo uma proporção similar a de humanos que ficam acordados 16 horas por dia
e dormem 8~\cite{waterhouseDaily2012}.

\section{Análise dos resultados}

\subsection{Formação de assembleias neuronais}

O número médio de neurônios por assembleia neuronal na RNP base foi de 787, enquanto na RNP com sono foi de 432.67; isso se deve
principalmente ao fato de que as assembleias neuronais do círculo e do quadrado tiveram pouquíssimos neurônios, apenas 42 e 34 na
RNP com sono. Ambas as RNPs tiveram dificuldades em manter uma assembleia neuronal única para o estímulo do círculo, havendo
bastante sobreposição com os neurônios da assembleia neuronal do quadrado; isso provavelmente ocorre pela similaridade entre os
dois estímulos e também a possibilidade da RNP ter alcançado um limite de memória. A matriz de sobreposição das assembleias
neuronais está ilustrada nas Figura~\ref{fig_mat_base}.

\begin{figure}[!ht]
\caption{Matrizes contendo o número de neurônios pertecentes a cada assembleia neuronal. Os elementos da diagonal principal da
matriz indicam o número total de neurônios em cada assembleia individual. Nos outros casos, o elemento representa quantos
neurônios de uma assembleia também são parte de outra assembleia.}

\centering{
\parbox{\linewidth}{
\includegraphics[width=7.5cm]{figuras/plots/mat_base.png}\label{fig_mat_base}
\hfill
\includegraphics[width=7.5cm]{figuras/plots/mat_sleep.png}\label{fig_mat_sleep}
\fonte{Elaborado pelo autor (2023).}}}
\end{figure}

Outra forma de analisar o comportamento das RNPs é verificando a ativação média das assembleias neuronais para cada estímulo, como
mostra a Figura~\ref{fig_mat_act_base}. A diagonal principal claramente possui as maiores médias pois a assembleia neuronal de um
determinado estímulo vai apresentar muito mais atividade para esse estímulo. Nota-se também que na RNP com sono houve mais
atividade nas assembleias neuronais não relacionadas com o estímulo, indicando uma piora na performance.

\begin{figure}[!ht]
\caption{Matrizes representando a ativação média em Hz de cada assembleia neuronal para cada estímulo diferente.}
\centering{
\parbox{\linewidth}{
\includegraphics[width=\linewidth]{figuras/plots/mat_act_base_sleep.png}\label{fig_mat_act_base}
\fonte{Elaborado pelo autor (2023).}}}
\end{figure}

Com base nesses resultados, pode-se deduzir que a simulação do sono não afetou significativamente a formação de assembleias
neuronais, ou que no máximo serviu apenas para limitar as capacidades da rede.

\subsection{Atividade da RNP}

Outra forma de analisar os resultados da simulação é analisando diretamente os dados dos disparos dos neurônios. A
Figura~\ref{fig_base_act} contém um gráfico relacionando o momento dos estímulos (topo da figura) com a ativação média das
assembleias neuronais (parte inferior da figura). É possível identificar que na maioria das vezes em que um estímulo é apresentado
à RNP, a assembleia neuronal associada a esse estímulo dispara e mantém-se ativa até que outro estímulo domine a atividade da
RNP;\@é importante notar que o estímulo é mostrado para a RNP por menos de 1 segundo, portanto a ativação da assembleia neuronal
nos segundos seguintes é a memória do estímulo.

Também pode-se notar que quando é mostrado o estímulo do círculo ou do quadrado, as assembleias neuronais de ambos os estímulos
disparam juntas, devido à sobreposição de neurônios entre as duas assembleias neuronais\footnote{Se é que podem ser consideradas
duas assembleias neuronais, já que os neurônios são quase todos os mesmos em ambas.} identificada na seção anterior.

\begin{figure}[!ht]
\caption{Ativação das assembleias neuronais na simulação da RNP base.}
\centering{
\parbox{\linewidth}{
\includegraphics[width=\linewidth]{figuras/plots/base_act.png}\label{fig_base_act}
\fonte{Elaborado pelo autor (2023).}}}
\end{figure}

\begin{figure}[!ht]
\caption{Ativação das assembleias neuronais na simulação da RNP com sono em momento de ``sonho''. A cor cinza indica o período de sono.}
\centering{
\parbox{\linewidth}{
\includegraphics[width=\linewidth]{figuras/plots/sleep_act2.png}\label{fig_sleep_act2}
\fonte{Elaborado pelo autor (2023).}}}
\end{figure}

\begin{figure}[!ht]
\caption{Ativação das assembleias neuronais na simulação da RNP com sono em outro momento de ``sonho''. A cor cinza indica o período de sono.}
\centering{
\parbox{\linewidth}{
\includegraphics[width=\linewidth]{figuras/plots/sleep_act.png}\label{fig_sleep_act}
\fonte{Elaborado pelo autor (2023).}}}
\end{figure}