\chapter{Metodologia}

Nesse capítulo serão apresentados os passos metodológicos para a realização do trabalho, que tem como objetivo principal estudar a
formação e consolidação de memórias em RNPs, com foco em analisar o impacto do sono nesse processo. Para estudar a influência do
sono na formação e recordação de assembleias neu\-ro\-nais será simulado um modelo de RNP com diferentes formas de plasticidade. A
Figura~\ref{fig_metodologia} apresenta uma visão geral da metodologia, contendo os passos para criação do modelo, que serão
descritos nas seções subsequentes.

\begin{figure}[!ht]
\caption{Visão geral da metodologia: passos para construção do modelo e simulação e análise.}
\centering{
\parbox{\linewidth}{
\includegraphics[width=\linewidth]{figuras/metodologia.png}\label{fig_metodologia}
% \fonte{}
}}
\end{figure}


\section{Modelo dos neurônios}

A unidade básica de uma RNP é o neurônio, então o primeiro passo para o modelo é a modelagem do neurônio. O modelo de neurônio
utilizado foi o \textit{Leaky Integrate-and-Fire} (LIF) devido à sua simplicidade, que captura o comportamento geral de um
neurônio enquanto permite simulações rápidas de larga escala, como apresentado na Seção~\ref{section_modelos_neuronios}.

O modelo LIF utilizado possui algumas diferenças do descrito na Seção~\ref{section_modelos_neuronios}, pois implementa adaptação
por frequência de disparo\footnote{Comportamento de um neurônio que reduz sua frequência de disparo em resposta a um estímulo
constante.} e segue a Equação~\ref{eq_lif_sfa}.

\begin{equation}
\label{eq_lif_sfa}
\tau^m\frac{dv_i}{dt} = (v^{rest} - v_i) + g_i^{exc}(t)(v^{exc} - v_i)+ (g_i^{gaba}(t) + g_i^{a}(t))(v^{ini} - v_i)
\end{equation}

\noindent{}onde $v_i(t)$, $v^{rest}$, $v^{exc}$, $v^{ini}$ se referem, respectivamente, ao potencial de membrana do neurônio,
potencial de repouso, potencial excitatório e potencial inibitório. As condutâncias são descritas por $g_i^{exc}(t)$,
$g_i^{gaba}(t)$, $g_i^{a}(t)$, respectivamente excitatória, inibitória pelos neurônios pré-sinápticos GABAérgicos e inibitória
pela adaptação por frequência de disparo, e o comportamento dessas condutâncias está definido nas
Equações~\ref{eq_lif_sfa_exc},~\ref{eq_lif_sfa_ini} e~\ref{eq_lif_sfa_a}. 

\begin{equation}
\label{eq_lif_sfa_exc}
\frac{dg_i^{exc}}{dt} = -\frac{g_i^{exc}}{\tau^{exc}} + \sum_{j\in exc}{w_{ij}S_j(t)}
\end{equation}

\begin{equation}
\label{eq_lif_sfa_ini}
\frac{dg_i^{gaba}}{dt} = -\frac{g_i^{gaba}}{\tau^{gaba}} + \sum_{j\in ini}{w_{ij}S_j(t)}
\end{equation}

\begin{equation}
\label{eq_lif_sfa_a}
\frac{dg_i^{a}}{dt} = -\frac{g_i^{a}}{\tau^{a}} + \Delta^{a}S_i(t)
\end{equation}

\begin{equation}
\label{eq_lif_sfa_spikes}
S_j(t) = \sum_{k}{\delta(t-t_j^k)}
\end{equation}

\noindent{}onde $w_{ij}$ refere-se ao peso da sinapse do neurônio $i$ para o $j$. A Equação~\ref{eq_lif_sfa_spikes} representa a
soma de disparos no momento $t$. Nessas equações, a condutância $g$ tende a zero com o tempo, mas quando há um disparo dos
neurônios pré-sinápticos essa condutância sobe, exceto no caso da condutância pela adaptação do neurônio, que aumenta seguindo um
fator $\Delta^{a}S_i(t)$ quando o próprio neurônio dispara.


\section{Modelo das sinapses}

O próximo passo foi simular como os neurônios interagem entre si. Para isso, foi utilizado o modelo de sinapse de condução,
apresentado na Seção~\ref{subsection_modelos_sinapses}.

\section{Modelos de plasticidade}

As sinapses, porém, não serão estáticas e terão diversas formas de plasticidade simuladas. Todas as sinapses serão plásticas, de
acordo com os seguintes modelos de plasticidade: PCP, PDTD-E, heterossináptica e a induzida por transmissor, apresentados na
Seção~\ref{section_modelos_plasticidade}.

\section{Arquitetura da rede}

A rede é estruturada com um total de 5.120 neurônios do tipo LIF, distribuídos em dois grupos principais. O primeiro grupo é
formado por 4.096 neurônios excitatórios, enquanto o segundo grupo contém 1.024 neurônios inibitórios. Além disso, há grupos
auxiliares como a retina (Detalhada na Seção~\ref{subsection_retina}) e o grupo responsável pelos padrões de sono (Detalhado na
Seção~\ref{subsection_experimento}). Para as conexões entre o grupo excitatório e inibitório, cada neurônio é conectado a 10\% dos
neurônios de outro grupo ou do seu próprio grupo, e essa conectividade é estabelecida de forma aleatória. As conexões excitatórias
entre neurônios são moduladas pelo neurotransmissor glutamato, com receptores AMPA e NMDA, enquanto as conexões inibitórias são
moduladas pelo neurotransmissor GABA.

\subsection{Modelo da retina}\label{subsection_retina}

De modo a simular a entrada de estímulos visuais, foi simulada uma retina. A retina é composta de 4.096 neurônios LIF, com cada
neurônio representando um pixel em uma imagem de 64$\times$64 pixels. Cada neurônio excitatório da RNP recebe conexões dos
neurônios da retina de uma área circular de raio 8, em que o centro do círculo é escolhido aleatoriamente para cada neurônio. As
conexões da retina com a RNP também são plásticas.

Como explicado na Seção~\ref{section_experimento}, os estímulos são imagens simples de serem reconhecidas, como formas
geométricas e símbolos. Essas imagens serão binárias: apenas preto e branco; o neurônio da retina correspondente a um pixel preto
irá disparar com frequência de 10Hz, enquanto o neurônio correspondente a um pixel branco irá disparar com frequência de 35Hz.

\subsection{Simulação do sono}\label{subsection_sono}

Para simular o sono, a rede funcionará de dois modos diferentes intercalados: um modo de atividade, em que a rede vai funcionar
normalmente enquanto recebe estímulos, e um modo de inatividade, em que será simulado o sono. 

Durante a fase de inatividade, nenhum estímulo será apresentado à rede, mas ela vai continuar sendo simulada normalmente. Além
disso, essa fase será dividida em duas subfases de acordo com as fases do sono real REM e NREM. 


Para simular cada fase do sono, existe um grupo de 256 neurônios que ativam de maneira sinusoidal e conectam-se a 20\% dos
neurônios do grupo de neurônios excitatórios

receberão uma corrente sinusoidal de frequência e amplitude diferentes para cada fase de modo a tentar imitar os padrões de
atividade cerebral durante o sono observados em eletroencefalograma; também serão simulados outros sinais característicos do sono
de forma similar injetando corrente, como os fusos do sono que ocorrem durante a fase N2.

\section{Experimentos}\label{section_experimento}

Como forma de avaliar o modelo proposto serão conduzidos dois principais experimentos.

\subsection{Experimento 1: Formação de assembleias neuronais}

O primeiro experimento consiste em simular a RNP apresentando estímulos ao modelo de retina e analisar se a repetição dos
estímulos leva à formação de assembleias neuronais associadas a cada estímulo a longo prazo, com o objetivo de ter uma base de
comparação para o experimento 2, que terá o sono simulado. Os estímulos consistem em seis imagens simples de serem reconhecidas,
exibidas na Figura~\ref{fig_estimulos}, e são apresentados à rede de forma intercalada e aleatória. Quatro estímulos foram
reaproveitados do trabalho de~\citeonline{zenkeDiverse2015}, enquanto as figuras de diamante e de cruz foram adicionadas com o
intuito de colocar a RNP mais próxima do seu limite.

\begin{figure}[!ht]
\caption{Os seis estímulos apresentados à RNP durante os experimentos.}
\centering{
\parbox{12cm}{
\includegraphics[width=12cm]{figuras/estimulos_cor.png}\label{fig_estimulos}
\fonte{Elaborado pelo autor (2023).}}}
\end{figure}

Houveram três fases da simulação da RNP:

\begin{enumerate}
  \item Simulação da rede em seu estado inicial por 1800 segundos, com um tempo médio de aparição do estímulo de 2 segundos, e
  tempo médio entre estímulos de 1 segundo. O peso das sinapses entre a retina e a rede é de 0.05.
  \item Simulação da rede também por 1800 segundos, mas com um tempo médio de aparição do estímulo de 0.2 segundo, e
  tempo médio entre estímulos de 5 segundos. O peso das sinapses entre a retina e a rede agora é de 0.1. A intenção aqui é
  fazer com que a rede tenha mais tempo entre um estímulo e outro para poder memorizá-los melhor.
  \item A última simulação é de 2400 segundos e é de onde são tirados os resultados. O tempo médio de aparição do estímulo diminui
  para 0.1, enquanto o tempo médio entre estímulos é de 10 segundos, com a intenção de ter uma janela maior de tempo entre os
  estímulos para analisar a capacidade de memorização da RNP.
\end{enumerate}


\subsection{Experimento 2: Formação de assembleias neuronais com sono}

De forma similar ao experimento 1, o segundo experimento consiste em simular a RNP apresentando os mesmos estímulos, mas dessa vez
com a simulação de sono. O objetivo desse experimento é analisar se a simulação de sono tem algum efeito na formação de
assembleias neuronais.

Esse experimento seguiu as mesmas três fases do experimento anterior, mas com a simulação do sono. Nesse experimento, a RNP ficava
em estado de vigília por 200s e dormia por 100s, seguindo uma proporção similar a de humanos que ficam acordados 16 horas por dia
e dormem 8~\cite{waterhouseDaily2012}.

\section{Análise dos resultados}

\subsection{Formação de assembleias neuronais}

O número médio de neurônios por assembleia neuronal na RNP base foi de 787, enquanto na RNP com sono foi de 432.67; isso se deve
principalmente ao fato de que as assembleias neuronais do círculo e do quadrado tiveram pouquíssimos neurônios, apenas 42 e 34 na
RNP com sono. Ambas as RNPs tiveram dificuldades em manter uma assembleia neuronal única para o estímulo do círculo, havendo
bastante sobreposição com os neurônios da assembleia neuronal do quadrado; isso provavelmente ocorre pela similaridade entre os
dois estímulos e também a possibilidade da RNP ter alcançado um limite de memória. A matriz de sobreposição das assembleias
neuronais está ilustrada nas Figura~\ref{fig_mat_base}.

\begin{figure}[!ht]
\caption{Matrizes contendo o número de neurônios pertecentes a cada assembleia neuronal. Os elementos da diagonal principal da
matriz indicam o número total de neurônios em cada assembleia individual. Nos outros casos, o elemento representa quantos
neurônios de uma assembleia também são parte de outra assembleia.}

\centering{
\parbox{\linewidth}{
\includegraphics[width=7.5cm]{figuras/plots/mat_base.png}\label{fig_mat_base}
\hfill
\includegraphics[width=7.5cm]{figuras/plots/mat_sleep.png}\label{fig_mat_sleep}
\fonte{Elaborado pelo autor (2023).}}}
\end{figure}

Outra forma de analisar o comportamento das RNPs é verificando a ativação média das assembleias neuronais para cada estímulo, como
mostra a Figura~\ref{fig_mat_act_base}. A diagonal principal claramente possui as maiores médias pois a assembleia neuronal de um
determinado estímulo vai apresentar muito mais atividade para esse estímulo. Nota-se também que na RNP com sono houve mais
atividade nas assembleias neuronais não relacionadas com o estímulo, indicando uma piora na performance.

\begin{figure}[!ht]
\caption{Matrizes representando a ativação média em Hz de cada assembleia neuronal para cada estímulo diferente.}
\centering{
\parbox{\linewidth}{
\includegraphics[width=\linewidth]{figuras/plots/mat_act_base_sleep.png}\label{fig_mat_act_base}
\fonte{Elaborado pelo autor (2023).}}}
\end{figure}

Com base nesses resultados, pode-se deduzir que a simulação do sono não afetou significativamente a formação de assembleias
neuronais, ou que no máximo serviu apenas para limitar as capacidades da rede.

\subsection{Atividade da RNP}

Outra forma de analisar os resultados da simulação é analisando diretamente os dados dos disparos dos neurônios. A
Figura~\ref{fig_base_act} contém um gráfico relacionando o momento dos estímulos (topo da figura) com a ativação média das
assembleias neuronais (parte inferior da figura). É possível identificar que na maioria das vezes em que um estímulo é apresentado
à RNP, a assembleia neuronal associada a esse estímulo dispara e mantém-se ativa até que outro estímulo domine a atividade da
RNP;\@é importante notar que o estímulo é mostrado para a RNP por menos de 1 segundo, portanto a ativação da assembleia neuronal
nos segundos seguintes é a memória do estímulo.

Também pode-se notar que quando é mostrado o estímulo do círculo ou do quadrado, as assembleias neuronais de ambos os estímulos
disparam juntas, devido à sobreposição de neurônios entre as duas assembleias neuronais\footnote{Se é que podem ser consideradas
duas assembleias neuronais, já que os neurônios são quase todos os mesmos em ambas.} identificada na seção anterior.

\begin{figure}[!ht]
\caption{Ativação das assembleias neuronais na simulação da RNP base.}
\centering{
\parbox{\linewidth}{
\includegraphics[width=\linewidth]{figuras/plots/base_act.png}\label{fig_base_act}
\fonte{Elaborado pelo autor (2023).}}}
\end{figure}

\begin{figure}[!ht]
\caption{Ativação das assembleias neuronais na simulação da RNP com sono em momento de ``sonho''. A cor cinza indica o período de sono.}
\centering{
\parbox{\linewidth}{
\includegraphics[width=\linewidth]{figuras/plots/sleep_act2.png}\label{fig_sleep_act2}
\fonte{Elaborado pelo autor (2023).}}}
\end{figure}

\begin{figure}[!ht]
\caption{Ativação das assembleias neuronais na simulação da RNP com sono em outro momento de ``sonho''. A cor cinza indica o período de sono.}
\centering{
\parbox{\linewidth}{
\includegraphics[width=\linewidth]{figuras/plots/sleep_act.png}\label{fig_sleep_act}
\fonte{Elaborado pelo autor (2023).}}}
\end{figure}


% https://sci-hub.mksa.top/10.1523/JNEUROSCI.22-19-08691.2002
% mencionar esse artigo que estimula 25% dos neurônios no sono
% \cite{bazhenovModel2002}
% !!!!!!!!!!!!!!!!!!!!!!!!!!!!!!!!!!!!!!!!!!!!!!!!!!!!!!!!!!!!!!!!!!!!!!!!!!!!!!!!!!