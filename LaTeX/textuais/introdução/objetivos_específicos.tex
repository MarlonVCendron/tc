\section{Objetivos específicos}

Para melhor entendimento do objetivo geral, os seguintes objetivos específicos são propostos:

\begin{itemize}

  \item Investigar as características e propriedades do sono para criar uma base sólida para a simulação do mesmo em uma RNP.\@

  \item Construir um modelo de RNP com simulação de sono, separando-o em sono REM e NREM.\@

  \item Sugerir e validar métodos para avaliar a consolidação e retenção de memórias na RNP a fim de comparar a performance da
  RNP com e sem a simulação do sono.

  \item  Analisar os resultados da simulação para identificar como a consolidação e retenção de memórias são afetadas pelas
diferentes fases do sono e pela atividade neural durante essas fases.

\end{itemize}

\section{Estrutura do trabalho}

O trabalho está organizado da seguinte maneira: o Capítulo~\ref{cap_fundamentacao} apresenta a base teórica do trabalho,
descrevendo conceitos de neurociência e neurociência computacional utilizados no trabalho; o Capítulo~\ref{cap_metodologia}
descreve a metodologia utilizada no trabalho para abordar o problema; o Capítulo~\ref{cap_experimentos} descreve os experimentos
conduzidos, bem como discute os resultados obtidos; e, por fim, o Capítulo~\ref{cap_conclusao} traz o fechamento do trabalho e as
considerações finais.
