\chapter{Introdução}

A busca pela compreensão e reprodução das habilidades cognitivas e de aprendizado do cérebro humano tem sido um desafio constante
nas áreas de neurociência computacional e Inteligência Artificial (IA). É possível argumentar que as Redes Neurais Artificiais
(RNA) são o mais próximo que já chegamos dessa reprodução; entretanto, as RNAs deixaram de lado o realismo biológico em prol do
aperfeiçoamento da IA~\cite{yamazakiSpiking2022}. As Redes Neurais de Disparos (RND)\footnote{Do inglês \textit{Spiking Neural
Networks}.} representam um avanço significativo em direção ao objetivo de compreender o cérebro humano, uma vez que buscam emular
o comportamento das Redes Neurais Biológicas (RNB) de forma mais realista do que as abordagens tradicionais.

As RNAs convencionais são inspiradas no cérebro: neurônios disparam em determinadas frequências conforme os sinais recebidos de
conexões com outros neurônios através de sinapses plásticas, cuja força muda dinamicamente de acordo com o treinamento.
Entretanto, as semelhanças com o cérebro estão limitadas a este ponto, uma vez que as RNAs tradicionais não capturam a dinâmica
interna dos neurônios biológicos, que disparam de maneiras complexas e distintas, e não apenas em uma determinada frequência.
Outra diferença entre as RNAs e os sistemas biológicos é que elas possuem um período de treinamento em que as sinapses são
otimizadas, e um período em que não há mais treinamento e as sinapses se tornam estáticas; enquanto que nas RNBs as sinapses estão
sempre se alterando conforme a experiência, salvo nos raros casos em que há um período crítico de aprendizado durante a infância,
que é desativado quando o indivíduo se torna adulto~\cite{crepelRegression1982}.

As RNDs são modelos muito mais próximos das RNBs que se comunicam por meio de impulsos elétricos discretos, chamados de disparos,
e que aprendem por métodos realistas, como a plasticidade das sinapses. O grau de realismo biológico de uma RND depende de sua
implementação, podendo empregar modelos de neurônios tão simples quanto uma única equação, que descreve a mudança de tensão
elétrica de um neurônio~\cite{burkittReview2006}, ou até modelos que simulam canais de íons~\cite{hodgkinQuantitative1952},
ramificações de dendritos~\cite{pagkalosIntroducing2023}, entre outros. As RNDs não só representam uma possível evolução das RNAs,
como também são usadas para seu propósito original: compreender o cérebro através da simulação.

A principal característica que torna as RNAs capazes de aprender é seu método de retropropagação de erro, um método de treinamento
que até pode existir em alguns casos no cérebro~\cite{lillicrapBackpropagation2020,songCan2020}, mas que é diferente da forma de
aprendizado local por plasticidade das RNBs~\cite{yamazakiSpiking2022}.

No entanto, treinar RNDs continua sendo uma tarefa desafiadora, já que os algoritmos de aprendizado empregados nas RNAs, além de
não serem biologicamente realistas, também não são diretamente aplicáveis às RNDs devido à natureza discreta dos disparos que as
torna não diferenciáveis, impedindo o cálculo de gradientes, parte fundamental no treinamento de RNAs.

O principal método empregado para o aprendizado de RNDs é a plasticidade das sinapses. A plasticidade é a capacidade do cérebro de
se adaptar e reorganizar suas conexões neurais em resposta a novas informações, experiências ou estímulos; é a principal
propriedade por trás do aprendizado e da formação de memórias. Uma das principais formas de plasticidade neural foi primeiramente
descrita por~\cite{hebbOrganization1949}, chamada de plasticidade hebbiana e influenciada pelas ideias
de~\cite{santiagoCroonian1894}, que pode resultar no fortalecimento ou enfraquecimento das sinapses com base na ativação
simultânea de neurônios conectados: caso o neurônio pós-sináptico dispare logo após o neurônio pré-sináptico, significa que há uma
correlação entre eles e a sinapse é fortalecida, caso contrário, a sinapse é enfraquecida.

A plasticidade hebbiana, resumida pela expressão ``neurônios que disparam juntos, co\-nectam-se juntos'', descreve a formação de
conjuntos celulares\footnote{Do inglês \textit{Cell Assemblies}. Também traduzido como Assembleias Celulares.} como resultado do
fortalecimento das conexões entre neurônios ativados simultaneamente. Esses conjuntos podem funcionar como mecanismos de memória
associativa~\cite{sakuraiMultiple2018}. Tomando como exemplo a memória de uma viagem à praia: essa memória consiste em vários
elementos, como o som das ondas, a sensação de areia sob os pés, o cheiro de água salgada, a visão do mar, entre outros. Cada um
desses elementos sensoriais é processado em diferentes áreas do cérebro e ativa diferentes grupos de neurônios. A ativação
síncrona dos neurônios responsáveis por esses elementos sensoriais leva à formação de um conjunto celular. Algum tempo depois, ao
sentir o cheiro do mar novamente, esse estímulo pode acabar ativando o conjunto celular, resultando na experiência da memória.
Conjuntos celulares não são estruturas estáticas, essas redes dinâmicas de neurônios que surgem a partir da experiência estão
sujeitas a modificações e reativações ao longo do tempo, sendo influenciadas pela falta de estímulos ou novas informações, o que
pode levar à alteração ou esquecimento de partes da memória.

A plasticidade hebbiana, no entanto, não consegue gerar, por si só, conjuntos celulares estáveis quando simulada em uma RND;\@
isso ocorre pois a atividade neural continuamente modifica as sinapses, fazendo com que em pouco tempo quaisquer estímulos não
relacionados com a informação codificada no conjunto celular acabem alterando as sinapses e desfazendo o conjunto
celular~\cite{gerstnerSpiking2002}.

Mas a plasticidade hebbiana não descreve toda a gama de diferentes modos com que a plasticidade se manifesta no cérebro, como é o
caso das  plasticidades heterossináptica, em que a ativação de neurônios causa mudanças em neurônios inativos, e homeostática, um
processo lento em que as sinapses se auto-regulam para garantir estabilidade. A plasticidade também depende do tipo de neurônio,
do tipo da conexão, do tempo de efeito das alterações (curto ou longo-prazo), entre outros fatores. A natureza do efeito da
plasticidade também varia muito, podendo depender da frequência de disparos, da diferença de potencial, do tempo dos disparos,
entre outros. Nas RNDs, assim como ocorre com os modelos de neurônios, os modelos de plasticidade também possuem uma ampla
variação em termos de plausibilidade biológica. Além disso, dependendo do modelo que se deseja utilizar, pode-se combinar
múltiplos modelos de plasticidade simultaneamente. Uma RND com plasticidade hebbiana junto de outras formas de plasticidade é
capaz de formar conjuntos celulares estáveis por horas~\cite{zenkeDiverse2015}.

O sono é um processo fisiológico crucial para a consolidação\footnote{A consolidação de uma memória é entendida como o processo
que transforma novas memórias frágeis criadas enquanto acordado para memórias mais estáveis e de longo prazo} e manutenção das
memórias~\cite{blissittSleep2001, walkerSleep2006, diekelmannMemory2010}. Inicialmente, postulava-se que o sono desempenhava uma
função passiva no processo de consolidação da memória~\cite{jenkinsObliviscence1924}; contudo, com a descoberta das distintas
fases do sono, começaram-se a explorar as contribuições ativas do sono na consolidação mnemônica~\cite{aserinskyRegularly1953}.
Durante o sono, ocorrem diferentes fases caracterizadas por padrões distintos de atividade cerebral: sono REM (\textit{Rapid Eye
Movement}) e sono não REM (NREM, dividido entre as fases N1, N2 e N3)~\cite{schulzRethinking2008}. Durante a fase NREM, oscilações
lentas, fusos e ondulações coordenam a reativação e redistribuição de memórias dependentes do hipocampo para o
neocórtex~\cite{diekelmannMemory2010}. Já quanto ao sono REM, a dificuldade em isolar a atividade neural dessa etapa específica,
que ocorre após a fase NREM, torna a discussão sobre sua contribuição para a consolidação da memória ainda controversa. Contudo,
pesquisas mais recentes oferecem evidências de que o sono REM desempenha um papel fundamental na consolidação da memória espacial
e contextual~\cite{boyceREM2017}.

Neste contexto, o problema a ser abordado neste trabalho consiste em explorar a retenção de memórias em uma RND.\@ A principal
hipótese a ser avaliada neste trabalho é de que abordagens baseadas em simulações de fases do sono podem melhorar a estabilidade
de conjuntos celulares contribuindo para o processo de retenção de memórias.

\section{Objetivo geral}

O objetivo geral consiste em desenvolver simulações das diferentes fases do sono em RNPs, analisando como a consolidação e
retenção de memórias da rede pode ser afetada.


\section{Objetivos específicos}

Para melhor entendimento do objetivo geral, os seguintes objetivos específicos são propostos:

\begin{itemize}

  \item Investigar as características e propriedades das diferentes fases do sono para criar uma base sólida para a simulação das
das mesmas em uma RND.\@

  \item Estudar e selecionar o modelo de RND mais apropriado para a simulação das fases do sono, levando em consideração a
capacidade de representar a atividade neural durante o sono e a flexibilidade para incorporar diferentes mecanismos de
consolidação de memória.

  \item Sugerir e validar métodos para avaliar a consolidação e retenção de memórias na RND a fim de comparar a performance da
  RND com e sem a simulação do sono.

  \item  Analisar os resultados da simulação para identificar como a consolidação e retenção de memórias são afetadas pelas
diferentes fases do sono e pela atividade neural durante essas fases.

  \item Contribuir para o entendimento dos mecanismos subjacentes aos processos de aprendizado e memória no cérebro, assim como
discutir possíveis implicações e aplicações nos campos de neurociência computacional e inteligência artificial.

\end{itemize}




\section{Estrutura do trabalho}

O trabalho está estruturado da seguinte forma: o Capítulo 2 apresenta a fundamentação teórica, o Capítulo 3 a metodologia e o
Capítulo 4 os resultados esperados, cronograma e considerações finais.


