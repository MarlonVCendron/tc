\chapter{Conclusão}\label{cap_conclusao}

Este estudo propôs-se a investigar a formação de assembleias neuronais em uma Rede Neural Pulsada (RNP) e os efeitos da simulação
de sono sobre esta formação.

Inicialmente foi conduzido um experimento onde observou-se que a RNP foi capaz de formar assembleias neuronais que correspondiam a
estímulos específicos. No entanto, limitações foram identificadas, particularmente na diferenciação entre estímulos similares,
como o círculo e o quadrado, indicando uma possível limitação na capacidade de memorização da rede considerando a arquitetura
proposta.

Em um segundo momento, foi conduzido um experimento incluindo a simulação de sono, que não apresentou melhora significativa na
formação das assembleias neuronais, indicando que os resultados apresentam-se inconclusivos. Curiosamente, as assembleias formadas
durante a simulação de sono foram menores e, de alguma forma, a atividade relacionada a estímulos não correspondentes foi maior, o
que sugere uma degradação na especificidade da resposta da rede. Isso pode indicar que o mecanismo de sono implementado não
reproduziu fielmente os benefícios observados no sono biológico, ou que a RNP tenha limitações inerentes que o sono simulado não
conseguiu superar.

A análise direta da atividade dos neurônios revelou padrões de reativação de assembleias neuronais durante o sono que são
reminiscentes do que se observa em organismos biológicos. Esses padrões poderiam ser analogamente interpretados como processos
similares aos sonhos em seres humanos e outros animais, com essas simulações podendo então servir como base de estudo para o
surgimento de tais fenômenos durante o sono biológico.

Em vista dos resultados apresentados, é evidente que a simulação do sono possui efeitos variados na formação de assembleias
neuronais em RNPs. No entanto, uma série de interrogações persiste e sinaliza caminhos para investigações futuras. Uma
possibilidade de investigação seria o uso dessa simulação em outros experimentos, como por exemplo avaliar a adaptabilidade da
rede ao aprender novos estímulos e esquecer os antigos, visto que a inter-relação entre sono e esquecimento ainda carece de um
entendimento mais profundo. Como a simulação do sono pode influenciar a atenuação ou perda de memórias neurais é uma questão que
merece atenção detalhada em estudos subsequentes. Além disso, a exploração de diferentes parâmetros na simulação do sono poderia
revelar dinâmicas de aprendizado e memória ainda não capturadas pelo presente estudo. Por fim, aproximar ainda mais as simulações
do sono das condições biológicas reais, poderá expandir a nossa compreensão sobre como esses estados alteram a plasticidade
sináptica e contribuem para a consolidação da memória em redes neurais artificiais. Estes passos subsequentes são fundamentais não
apenas para avançar no conhecimento científico da neurociência computacional mas também para explorar implicações práticas na
otimização de algoritmos de inteligência artificial e no desenvolvimento de novas arquiteturas de redes neurais.
