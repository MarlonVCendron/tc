\section{Considerações finais}

Ao longo deste trabalho, foram explorados os aspectos teóricos do desenvolvimento e da aplicação das Redes Neurais Pulsadas
(RNPs) para entender melhor a formação e consolidação de memórias no cérebro humano. A principal questão abordada foi a
possibilidade de simulações de fases do sono melhorarem a estabilidade e formação de assembleias neuronais, contribuindo para a
retenção de memórias. 

No entanto, este trabalho se concentrou na parte teórica dessas questões. O próximo passo, na continuação desse trabalho, será
implementar e aplicar um modelo que empregue as ideias discutidas aqui e analisar as implicações de simulações do sono em tal
modelo neural.

Por fim, vale salientar que este trabalho posiciona-se na intersecção crucial entre a neurociência computacional e a inteligência
artificial. Ao iluminar os processos subjacentes à formação e consolidação da memória no cérebro humano e o papel do sono nisso,
estamos expandindo a fronteira de nossa compreensão na neurociência. Além disso, ao melhorar nossa capacidade de simular esses
processos em modelos de RNPs, também estamos avançando na área de inteligência artificial, aprofundando nossa compreensão de como
a inteligência pode ser replicada e potencialmente aperfeiçoada. 

