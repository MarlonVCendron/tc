\section{Plasticidade}\label{section_plasticidade}

Plasticidade refere-se à capacidade do sistema nervoso de se modificar em resposta a experiências, estímulos e mudanças
ambientais. Esta adaptabilidade inerente é a base de vários processos cognitivos, incluindo a memória e a aprendizagem. No nível
celular, a plasticidade manifesta-se como mudanças na força das sinapses. Estas mudanças são mediadas por uma variedade de
processos moleculares e celulares que resultam na formação ou eliminação e fortalecimento ou enfraquecimento de sinapses.

A força de uma sinapse é definida como a influência que uma sinapse tem no potencial de membrana do neurônio pós-sináptico. Quanto
maior a força, maior a influência da sinapse em gerar um potencial de ação. De forma mais rigorosa, a força sináptica pode ser
definida como uma combinação de diversos fatores: a probabilidade de liberação de neurotransmissor
\footnote{Como mencionado na seção~\ref{sec_neurônios}, esse trabalho foca apenas nas sinapses químicas}
pré-sináptica, de quão relevante é a resposta pós-sináptica à liberação de uma única vesícula de neurotransmissor e ao número de
locais de liberação de neurotransmissor~\cite{fattSpontaneous1952}. 

Existem várias formas de plasticidade, a primeira a ser observada, a Potenciação de Longa Duração (PLD), foi descoberta
por~\cite{blissLonglasting1973}, quando a ativação repetitiva de sinapses excitatórias em neurônios do hipocampo de coelhos causou
um aumento na força sináptica que durava por horas ou até dias. A PLD é específica para cada estímulo, ou seja, quando gerada em
um conjunto de sinapses por ativação repetitiva, o aumento na força sináptica normalmente ocorre apenas em uma sinapse específica,
e não em outras sinapses na mesma célula. Em oposição à PLD, a Depressão de Longa Duração (DLD) é uma diminuição na força
sináptica que ocorre quando a ativação repetitiva de sinapses falha em causar um potencial de ação no neurônio pós-sináptico
consistentemente~\cite{dudekHomosynaptic1992}. Tanto a PLD e a DLD são exemplos de plasticidade hebbiana, ou seja, dependem da
coincidência temporal entre a ativação pré e pós-sináptica, e ambas são exemplos de Plasticidade Dependente do Tempo de Disparo
(PDTP).

A Plasticidade de Curto Prazo (PCP) é outra forma de plasticidade que envolve mudanças rápidas, mas transitórias, na força
sináptica em resposta a atividade neuronal. Ela pode se manifestar como facilitação ou depressão sináptica. A facilitação
sináptica de curto prazo ocorre quando a ativação repetitiva de uma sinapse aumenta temporariamente a sua eficácia, normalmente
dentro do intervalo de milissegundos a, no máximo, alguns minutos, como uma resposta ao aumento temporário na liberação de
neurotransmissores da célula pré-sináptica. Já a depressão sináptica de curto prazo é uma diminuição transitória na eficácia
sináptica, que ocorre quando a liberação de neurotransmissores é temporariamente reduzida em resposta a uma ativação sináptica
frequente ou sustentada~\cite{zuckerShortTerm2002}.

A plasticidade sináptica também pode ser classificada em dois tipos principais: heterossináptica e homossináptica. A plasticidade
heterossináptica refere-se àquelas contribuições para a plasticidade sináptica que dependem apenas do estado do neurônio
pós-sináptico, mas não do estado dos neurônios pré-sinápticos. Em contraste, a plasticidade homossináptica depende conjuntamente
da atividade pré e pós-sináptica. Ambos os tipos de plasticidade são essenciais para a adaptabilidade do sistema nervoso e para a
formação de memórias. No entanto, a plasticidade homossináptica é geralmente considerada a forma dominante de plasticidade no
aprendizado e memória devido ao seu papel na codificação da atividade conjunta de neurônios pré e pós-sinápticos, o que é uma
característica fundamental do processo de aprendizado~\cite{grangerExpression2014, feldmanSynaptic2009}.

Por fim, a plasticidade é uma característica multifacetada e dinâmica do sistema nervoso. Há várias outras formas de plasticidade
além das mencionadas, cada uma atuando de maneira distinta e sendo desencadeada por diferentes motivos, como a frequência de
ativação, diferença de tensão, entre outros, dependendo do neurônio específico, da sinapse e dos mecanismos moleculares e celulares
envolvidos.

