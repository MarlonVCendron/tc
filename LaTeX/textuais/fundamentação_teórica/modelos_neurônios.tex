\section{Modelos de Neurônios}\label{section_modelos_neuronios} 

Em 1907, Lapicque desenvolveu um modelo de neurônio que descreve o neurônio como um circuito elétrico contendo um capacitor e um
resistor em paralelo, como representado na Figura~\ref{fig_lapicque}, representando a capacitância e a resistência de vazamento da
membrana celular~\cite{lapicqueRecherches1907}, chamado de modelo Integrate-and-Fire (IF). Mesmo sem entender os mecanismos por
trás da geração de potenciais de ação, Lapicque postulou que, ao atingir um certo potencial limiar, um potencial de ação seria
gerado e o capacitor descarregado, reiniciando o potencial da membrana. Isso mostra que, ao se tratar de modelagem de neurônios,
estudos da função não necessariamente requerem conhecimento do mecanismo~\cite{abbottLapicque1999}. O modelo de neurônio de
Lapicque foi a primeira tentativa de representar matematicamente um neurônio biológico.

\begin{figure}[!ht]
\caption{(A) O circuito elétrico de Lapicque: I é a corrente injetada, C a capacitância da membrana, R a resistência da membrana,
V o potencial de membrana e $V_{rest}$ o potencial de repouso. (B) A trajetória de tensão, quando um limiar é atingido, um
potencial de ação é disparado. (C) Um modelo IF com corrente que varia pelo tempo.}
\centering{
\parbox{\linewidth}{
\includegraphics[width=\linewidth]{figuras/lapicque.png}\label{fig_lapicque}
\fonte{\citeonline{abbottLapicque1999}.}}}
\end{figure}

O modelo de Hodgkin-Huxley~\cite{hodgkinQuantitative1952}, representou um avanço significativo na compreensão e na modelagem dos
neurônios biológicos. Diferentemente do primeiro modelo criado por Lapicque, esse modelo buscou descrever a geração e propagação
de potenciais de ação em neurônios levando em consideração os processos eletroquímicos subjacentes, como a dinâmica dos diferentes
canais iônicos que controlam a corrente elétrica através da membrana celular.

O modelo de Hodgkin-Huxley é composto por um conjunto de equações diferenciais ordinárias que descrevem a variação do potencial de
membrana em função do tempo e das correntes iônicas. Essas equações consideram o comportamento dinâmico dos canais iônicos de
sódio e potássio, bem como a corrente de vazamento através da membrana. O modelo é capaz de capturar o comportamento típico dos
neurônios, incluindo a resposta ao estímulo, a fase refratária e a propagação do sinal ao longo do axônio.

Baseados na ideia de Lapicque, hoje em dia são utilizados os modelos IF.\@ Existem diversos modelos IF, com
várias modificações da ideia original, como o modelo Leaky Integrate-and-Fire (LIF)~\cite{burkittReview2006}. Os modelos IF são
uma alternativa mais simples e computacionalmente eficiente em comparação ao modelo de Hodgkin-Huxley. Embora não sejam tão
biologicamente precisos quanto o modelo de Hodgkin-Huxley, os modelos IF conseguem capturar algumas das características essenciais
dos neurônios, como a integração temporal dos estímulos e a emissão de potenciais de ação quando um limiar é atingido. Outra
vantagem dos modelos IF é que modelos simples são uma forma de reduzir a complexidade do cérebro para seus mecanismos mais
fundamentais.

Devido a sua simplicidade, os modelos IF têm sido amplamente utilizados em RNP pela sua eficiência computacional e capacidade de
reproduzir aspectos fundamentais do comportamento neuronal. Por exemplo, no trabalho de~\citeonline{teeterGeneralized2018} o
comportamento de 645 neurônios do neocórtex foi reproduzido utilizando modelos GLIF (Generalized Leaky Integrate-and-Fire). 

O modelo de neurônio LIF é descrito pela dinâmica do potencial de membrana do neurônio, $v(t)$, que é dado pela
Equação~\ref{eq_lif} e uma condição adicional para a geração de potenciais de ação, dada pela Equação~\ref{eq_lif_cond}:

\begin{equation}
\label{eq_lif}
C_m \frac{dv(t)}{dt} = -\frac{C_m}{\tau_m} [v(t) - V_0] + I(t)
\end{equation}

\begin{equation}
\label{eq_lif_cond}
\text{se}\quad v \ge v_{th} \quad \text{então} \quad v \gets v_{reset}
\end{equation}

\noindent{}onde $C_m$ é a capacitância da membrana, $V_0$ é o potencial de repouso, $\tau_m$ é a constante de tempo passiva da membrana
(relacionada à capacitância do neurônio e à resistência de vazamento do potencial de membrana por $\tau_m = R_m C_m$), $I(t)$ é a
corrente elétrica injetada no neurônio (tanto a corrente causada pelas sinapses, como a por eletrodos)~\cite{burkittReview2006}. 

Quando o potencial de membrana atinge um limiar, $V_{th}$, um potencial de ação é disparado e o potencial de membrana é resetado
para $V_{reset}$, o potencial um pouco menor que o potencial de repouso, correspondente ao período refratário do neurônio.

Existem diversos outros modelos de neurônios, como o modelo de Izhikevich~\cite{izhikevichSimple2003}, que é quase tão simples em
termos computacionais quanto o modelo IF, mas que consegue capturar de forma muito mais realista um conjunto maior de
comportamentos neurais dependendo dos parâmetros utilizados, como mostra a Figura~\ref{fig_izhikevich}.

\begin{figure}[!ht]
\caption{Diferentes tipos de neurônios simulados pelo modelo de Izhikevich.}
\centering{
\parbox{\linewidth}{
\includegraphics[width=\linewidth]{figuras/izhikevich.png}\label{fig_izhikevich}
\fonte{\citeonline{izhikevichSimple2003}.}}}
\end{figure}

O modelo de Izhikevich é especialmente útil quando se deseja estudar populações de neurônios específicos do cérebro. Já os modelos
IF, como o modelo LIF escolhido para ser utilizado nesse trabalho, são preferidos quando o objetivo é estudar o comportamento
geral de neurônios por conta de sua simplicidade.

\subsection{Modelos de Sinapses}\label{subsection_modelos_sinapses}

Para modelar a comunicação entre neurônios, é necessário modelar as sinapses. Existem diversos modelos de sinapses, que variam em
complexidade e precisão. O modelo mais simples e muito utilizado é o modelo de sinapse de condução, que é uma sinapse estática,
que não possui plasticidade e não se adapta ao longo do tempo. A dinâmica de condutância\footnote{A condutância é o inverso da
resistência.} da sinapse é descrita pela Equação~\ref{eq_sinapse}:

\begin{equation}
\label{eq_sinapse}
\frac{dg_{syn}(t)}{dt} = \bar{g}_{syn}\sum_{k}{\delta(t-t_k)} - g_{syn}(t)/\tau_{syn}
\end{equation}

\noindent{}onde $g_{syn}(t)$ refere-se à condutância da sinapse, $\bar{g}_{syn}$ é a condutância máxima da sinapse, ou o peso da
sinapse, que determina o quão forte é a influência da sinapse no neurônio pós-sináptico, $\delta(x)$ é a função delta de Dirac, que
vale 1 quando $x=0$ e 0 caso contrário, esse somatório resulta em 0 caso não tenha havido nenhum potencial de ação na sinapse no
tempo $t$.

A lei de Ohm\footnote{Dada por $V=IR$, onde a tensão é igual à corrente elétrica multiplicada pela resistência.} é utilizada para
calcular a corrente elétrica a partir da condutância da sinapse, que é dada pela Equação~\ref{eq_sinapse_curr}:

\begin{equation}
\label{eq_sinapse_curr}
I_{syn}(t)=g_{syn}(t)(V(t)-E_{syn})
\end{equation}

\noindent{}onde $V(t)$ é o potencial da membrana e $E_{syn}$ corresponde ao potencial de reversão da sinapse, que determina se a sinapse é
excitatória ou inibitória.

