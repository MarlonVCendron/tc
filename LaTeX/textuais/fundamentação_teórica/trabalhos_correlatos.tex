\section{Trabalhos correlatos}

A popularidade do uso de modelos computacionais de neurônios e RNPs para o estudo das funções cognitivas cerebrais reside em sua
capacidade de proporcionar um controle completo e preciso sobre todas as características de cada neurônio simulado. Diferentemente
dos neurônios biológicos em laboratório ou em seres humanos vivos, onde a manipulação precisa e a observação das variáveis são
intrinsecamente limitadas devido a complexidade biológica e ética, as RNPs permitem controle total sobre todos os parâmetros da
rede e de cada neurônio simulado, o que habilita a realização de experimentos teóricos detalhados, facilita o entendimento dos
mecanismos neurais subjacentes às funções cognitivas e acelera o desenvolvimento de aplicações práticas, como a melhoria dos
algoritmos de inteligência artificial. 

No trabalho de~\citeonline{zenkeDiverse2015}, os autores criaram uma RNP composta de 4.096 neurônios excitatórios e 1.024
neurônios inibitórios para testar a hipótese de que diferentes formas de plasticidade, tanto hebbianas como não hebbianas, quando
implementadas juntas poderiam levar à formação e à recordação de assembleias neuronais. O modelo criado pôde criar memórias dos
quatro diferentes estímulos visuais que recebia e essas memórias eram estáveis e podiam ser recordadas até mesmo depois de horas
de simulação sem que o modelo visse novamente os estímulos. Esse modelo de RNP serviu para demonstrar que é possível, pelo menos
em teoria, ter a formação de assembleias neuronais, e consequentemente memória, apenas a partir de mecanismos de plasticidade
orquestrados. Porém, algo que os autores não exploraram, ou pelo menos não tornaram explícito em seu trabalho, é a quantidade de
estímulos que a RNP é capaz de memorizar; também no mesmo tema, não é explorado o que acontece com o modelo quando um novo
estímulo é apresentado ao depois da memorização dos demais.

No trabalho de~\citeonline{goldenSleep2022}, os autores utilizam RNPs para treinar um agente a buscar alimentos em um ambiente
simulado. Ao tentar treinar a RNP em duas tarefas diferentes\footnote{Em cada tarefa, o tipo de alimento que o agente deveria
comer mudava de forma e ele deveria aprender a reconhecer essa forma no ambiente.}, a rede experienciava um fenômeno de
``esquecimento catastrófico'', em que uma das tarefas aprendidas era completamente esquecida após aprender a
outra~\cite{hasselmoAvoiding2017}. Ao implementar um estado de sono na rede, caracterizado por períodos muito curtos em que a rede
não recebia nenhum estímulo, o agente era incapaz de se mover, o tipo de plasticidade era trocado e também era inserido ruído em
cada neurônio, a rede se tornou capaz de aprender ambas as tarefas sem esquecer do que já havia aprendido anteriormente. Essa
simulação do sono foi feita para tentar imitar o sono REM ao se adicionar ruído, mas as outras fases do sono não foram simuladas.
Além disso, o modelo da RNP não era recorrente, como no cérebro humano, e era utilizado em uma tarefa de aprendizado de reforço, o
que não está no escopo deste trabalho.

No estudo de~\citeonline{bazhenovModel2002}, os pesquisadores criaram um modelo computacional sofisticado do sistema
corticotálamo, explorando as oscilações de ondas lentas durante o sono NREM e suas transições para estados mais ativos. Utilizando
uma combinação de gravações intracelulares in vivo e simulações computacionais, o modelo detalha a dinâmica entre estados
silenciosos e ativos, típicos do sono NREM em gatos não anestesiados. A descoberta central do estudo é que a acumulação ocasional
de potenciais pós-sinápticos excitatórios miniatura durante a fase hiperpolarizada do sono NREM pode desencadear uma corrente
persistente de sódio, levando à depolarização das células piramidais corticais e à geração de disparos. Este trabalho exemplifica
o poder da modelagem computacional em neurociência, demonstrando como simulações detalhadas e dados experimentais podem ser
integrados para descobrir mecanismos neuronais complexos.

No trabalho de~\citeonline{renno-costaComputational2019}, os autores abordam a complexidade das teorias sobre o papel do sono na
consolidação da memória através de modelos computacionais. O estudo enfatiza a importância do sono para alterações duradouras na
conectividade sináptica, fundamentais para o armazenamento de memórias. Os modelos explorados refletem as diversas teorias sobre o
sono e a plasticidade sináptica de longo prazo. Alguns modelos sugerem que o sono ajuda no esquecimento de memórias irrelevantes,
enquanto outros indicam seu papel no reforço das memórias mais valiosas. O estudo abrange teorias como a hipótese de homeostase
sináptica, que postula o enfraquecimento sináptico durante o sono, e a teoria de reativação, que propõe a consolidação de certos
padrões sinápticos. Também aborda os diferentes papéis dos estados de sono REM e NREM.\@ Cada modelo oferece suporte a diferentes
aspectos dessas teorias, destacando a complexidade e a multifuncionalidade do sono na consolidação da memória.

