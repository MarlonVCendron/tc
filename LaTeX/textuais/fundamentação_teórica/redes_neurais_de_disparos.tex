\section{Redes Neurais de Disparo}

Juntando os modelos de neurônios e sinapses descritos nas Seções~\ref{section_modelos_neuronios}
e~\ref{subsection_modelos_sinapses}, podemos construir modelos de Redes Neurais de Disparo (RNDs). As RNDs são modelos
computacionais que buscam emular a forma como os neurônios biológicos interagem e se comunicam entre si no
cérebro~\cite{yamazakiSpiking2022}. 

Para alguns pesquisadores, as RNDs hoje são vistas como uma terceira geração de redes neurais
artificiais~\cite{maassNetworks1997}. Atualmente, existe um intenso estudo sobre a aplicação das RNDs para solucionar problemas
típicos enfrentados pelas redes neurais artificiais, incluindo questões de visão computacional e classificação. A expectativa é de
que o realismo biológico presente nestes modelos possa impulsionar o avanço no campo da inteligência
artificial~\cite{yamazakiSpiking2022}.

Já em outras pesquisas, como é o caso desse trabalho, as RNDs são utilizadas como modelos para estudar o comportamento de
neurônios biológicos.

