%%%%%%%%%%%%%%%%%%%%%%%%%%%%%%%%%%%%%%%%%%%%%%%%%%%%%%%%%%%%%%%%%%%%%%%%%%%%
%%% 				Customizações do abnTeX2 para o IFCE    			 %%%
%%% 		Instituto Federal de Educação, Ciência e Tecnologia Catarinense %%%
%%%																		 %%%
%%% Template disponível em: https://github.com/clodomirneto/IFCETeX2	 %%%
%%% Desenvolvedores do IFCETeX2: Professor Clodomir Silva Lima Neto		 %%%
%%% 							 Professor Marcelo Araújo Lima			 %%%
%%% E-mails para contato: clodomir.neto@ifce.edu.br						 %%%
%%% 					  marcelo.alima@ifce.edu.br						 %%%
%%%																		 %%%
%%% Agradecimento: Thiago Nascimento 									 %%%
%%% https://github.com/thiagodnf/uecetex2                                %%%
%%%%%%%%%%%%%%%%%%%%%%%%%%%%%%%%%%%%%%%%%%%%%%%%%%%%%%%%%%%%%%%%%%%%%%%%%%%%

%%%%%%%%%%%%%%%%%%%%%%%%%%%%%%%%%%%%%%%%%%%%%%%%%%%%%%%%%%%%%%%%%%%%%%%%%%%%
%%% 						Sequência do GitHub							 %%%
%%%																	     %%%
%%% git status + git add . + git commit + git push					     %%%
%%%%%%%%%%%%%%%%%%%%%%%%%%%%%%%%%%%%%%%%%%%%%%%%%%%%%%%%%%%%%%%%%%%%%%%%%%%%

\documentclass[        
    a4paper,          % Tamanho da folha A4
    12pt,             % Tamanho da fonte 12pt
    chapter=TITLE,    % Todos os capitulos devem ter caixa alta
    %section=TITLE,   % Todas as secoes devem ter caixa alta
    oneside,          % Usada para impressao em apenas uma face do papel
    english,          % Hifenizacoes em ingles
    spanish,          % Hifenizacoes em espanhol
    brazil            % Ultimo idioma eh o idioma padrao do documento
]{abntex2}

% Importações de pacotes
\usepackage[utf8]{inputenc}                         % Acentuação direta
\usepackage[T1]{fontenc}                            % Codificação da fonte em 8 bits
\usepackage{graphicx}                               % Inserir figuras
\usepackage{amsfonts,amssymb,amsmath}               % Fonte e símbolos matemáticos
\usepackage{booktabs}                               % Comandos para tabelas
\usepackage{verbatim}                               % Texto é interpretado como escrito no documento
\usepackage{multirow,array}                         % Múltiplas linhas e colunas em tabelas
\usepackage{indentfirst}                            % Endenta o primeiro parágrafo de cada seção.
\usepackage{listings}                               % Utilizar codigo fonte no documento
\usepackage{xcolor}
% \usepackage{microtype}                              % Para melhorias de justificação?
\usepackage[portuguese,ruled,lined]{algorithm2e}    % Escrever algoritmos
\usepackage{algorithmic}                            % Criar Algoritmos  
%\usepackage{float}                                 % Utilizado para criação de floats
\usepackage{amsgen}
\usepackage{lipsum}                                 % Usar a simulação de texto Lorem Ipsum
%\usepackage{titlesec}                               % Permite alterar os títulos do documento
\usepackage{tocloft}                                % Permite alterar a formatação do Sumário
\usepackage{etoolbox}                               % Usado para alterar a fonte da Section no Sumário
%\usepackage[nogroupskip,nonumberlist,acronym]{glossaries}                % Permite fazer o glossario
\usepackage{caption}                                % Altera o comportamento da tag caption
\usepackage[alf,abnt-emphasize=bf,bibjustif, recuo=0cm,abnt-etal-cite=3,abnt-etal-list=0,abnt-etal-text=it]{abntex2cite}  % Citações padrão ABNT
%\usepackage[bottom]{footmisc}                      % Mantém as notas de rodapé sempre na mesma posição
%\usepackage{times}                                 % Usa a fonte Times
\usepackage{mathptmx}                               % Usa a fonte Times New Roman							
%\usepackage{lmodern}                               % Usa a fonte Latin Modern
%\usepackage{subfig}                                % Posicionamento de figuras
%\usepackage{scalefnt}                              % Permite redimensionar tamanho da fonte
%\usepackage{color, colortbl}                       % Comandos de cores
%\usepackage{lscape}                                % Permite páginas em modo "paisagem"
%\usepackage{ae, aecompl}                           % Fontes de alta qualidade
%\usepackage{picinpar}                              % Dispor imagens em parágrafos
%\usepackage{latexsym}                              % Símbolos matemáticos
%\usepackage{upgreek}                               % Fonte letras gregas
\usepackage{appendix}                               % Gerar o apendice no final do documento
\usepackage{paracol}                                % Criar paragrafos sem identacao
\usepackage{ifcetex2}		                        % Biblioteca com as normas do IFCE para trabalhos academicos
\usepackage{pdfpages}                               % Incluir pdf no documento
\usepackage{textcomp}
% Organiza e gera a lista de abreviaturas, simbolos e glossario
%\makeglossaries
% Gera o Indice do documento
%\makeindex

%TEOREMAS

\newtheorem{proposition}{Proposição}[chapter]
\newtheorem{theorem}{Teorema}[chapter]
\newtheorem{lemma}{Lema}[chapter]
\newtheorem{definition}{Definição}[chapter]
\newtheorem{exemplo}{Exemplo}[chapter]
\newtheorem{corollary}{Corolário}[chapter]
\newtheorem{exercicio}{Exercício}[chapter]

%PROVAS

\newenvironment{prova}[1][Prova]{\noindent\textbf{#1.} }{\hfill\rule{0.5em}{0.5em}}
\newenvironment{dem}[1][Demonstra\c c\~ao]{\noindent\textbf{#1.} }{\hfill\rule{0.5em}{0.5em}}
\newenvironment{exm}{\noindent{\textbf{Exemplo:}}}{}