\chapter{Fundamentação teórica}

\section{Neurônios biológicos}

Neurônios recebem sinais dos dendritos, esses sinais são integrados no soma e, se for o caso, um potencial de ação é disparado
pelo axônio. O axônio carrega o sinal que é transmitido para os dendritos de outros neurônios através da sinapse.

Sinapses podem ser químicas (através de neurotransmissores), mais lentas mas que podem intesificar o sinal, ou elétricas, rápidas
mas que não podem mudar a amplitude do sinal. Sinapses podem ser excitatórias, que despolarizam a célula pós-sináptica, ou
inibitórias, que hiperpolarizam. Os neurotransmissores excitatórios mais comuns são: glutamato, dopamina e noradrenalina. Os
inibitórios são: GABA, glicina e serotonina.

O potencial de membrana é o potencial elétrico interior da célula em relação ao exterior. Em neurônios, o potencial de repouso é
de -70,15 mV. Quando ocorre um potencial de ação, canais de Na+ são abertos até que o potencial de reversão do sódio é alcançado,
em 38,43 mV. O potencial de membrana cai até chegar no potencial de reversão do K+ e hiperpolariza um pouco além do potencial de
repouso, para então voltar ao potencial de repouso. Diferentes neurônios têm potencial de repouso e de reversão de sódio e
potássio diferentes, entre mais ou menos 10 mV.

// mais detalhes de neurônios
{...}

\section{Modelos de Neurônios}

!!! Na verdade Lapicque já tinha criado um modelo matemático em 1907. Modelo IF

Lapicque L (1907) Quantitative investigations of electrical
nerve excitation treated as polarization (in French). J
Physiol Pathol Gen 9: 620–635.

O modelo de neurônio de McCulloh-Pitts \cite{mccullochLogical1943} representou a primeira tentativa de representar matematicamente
um neurônio biológico. O modelo é bastante simplificado: o neurônio recebe X sinais de entrada binários, que podem ser
excitatórios (+1) ou inibitórios (-1). Esses sinais são integrados e, em seguida, é verificado se um limiar é atingido. Apenas
quando o limiar é ultrapassado, o neurônio emite um sinal. O modelo foi uma ideia inovadora ao tentar analisar a biologia por meio
de termos matemáticos. No entanto, o modelo não conseguiu capturar toda a complexidade de um neurônio biológico, pois essa não era
a intenção do trabalho. O objetivo principal era mostrar que era possível representar e analisar o sistema nervoso usando lógica
proposicional.

<equação>
<imagem mcculloh-pitts neuron>

O modelo de Hodgkin-Huxley \cite{hodgkinQuantitative1952}, representou um avanço significativo na compreensão e na modelagem dos
neurônios biológicos. Esse modelo buscou descrever a geração e propagação de potenciais de ação em neurônios, levando em
consideração os processos eletroquímicos subjacentes, como a dinâmica dos canais iônicos e as correntes iônicas que atravessam a
membrana celular. Diferentemente do modelo de McCulloch-Pitts, o modelo de Hodgkin-Huxley é capaz de descrever disparos de
potenciais de ação, fornecendo uma representação mais precisa e detalhada da atividade neuronal.

O modelo de Hodgkin-Huxley é composto por um conjunto de equações diferenciais ordinárias que descrevem a variação do potencial de
membrana em função do tempo e das correntes iônicas. Essas equações consideram o comportamento dinâmico dos canais iônicos de
sódio e potássio, bem como a corrente de vazamento através da membrana. O modelo é capaz de capturar o comportamento típico dos
neurônios, incluindo a resposta ao estímulo, a fase refratária e a propagação do sinal ao longo do axônio.

<equações>
<imagem comparando tensão de neurônio biológico e de Hodgkin-Huxley>

citar Stein 1967

Os modelos de neurônios do tipo Integrate-and-Fire (IF), como o modelo Leaky Integrate-and-Fire (LIF) \cite{burkitt2006review},
surgiram como uma alternativa mais simples e computacionalmente eficiente em comparação ao modelo de Hodgkin-Huxley. Embora não
sejam tão biologicamente precisos quanto o modelo de Hodgkin-Huxley, os modelos IF conseguem capturar algumas das características
essenciais dos neurônios, como a integração temporal dos estímulos e a emissão de potenciais de ação quando um limiar é atingido.
Outra vantagem dos modelos IF é que modelos simples são uma forma de reduzir a complexidade do cérebro para seus mecanismos mais
fundamentais.

<equação>

\section{Modelos de Plasticidade}

A plasticidade é um conceito chave na neurociência e se refere à capacidade dos neurônios de modificar suas sinapses e
comportamento em resposta a experiências e estímulos. Modelos de plasticidade incluem:

STP, STP, STDP, Homeostática, etc.

\section{Conjuntos Celulares}

A plasticidade hebbiana pode ser sumarizada pela seguinte frase "Neurônios que disparam juntos, conectam-se juntos", assim, a
plasticidade dá origem a um fenômeno emergente no cérebro chamado de conjuntos celulares\footnote{Do inglês \textit{Cell
Assemblies}. Também traduzido como Assembleias Celulares.}, em que grupos de neurônios relacionados a um mesmo estímulo ou
processo acabam fortalecendo as conexões entre si. Um conjunto celular pode servir como uma forma de memória associativa
\cite{sakuraiMultiple2018}, em que a ativação de um dos neurônios do conjunto acaba por ativar sincronamemente os demais neurônios
do conjunto devido ao fortalecimento das conexões entre os neurônios do conjunto. Tomando como exemplo a memória de uma viagem à
praia: essa memória consiste em vários elementos, como o som das ondas, a sensação de areia quente, o cheiro de água salgada e a
visão de gaivotas etc. Cada um desses elementos sensoriais é processado em diferentes áreas do cérebro e ativa diferentes grupos
de neurônios. A ativação síncrona dos neurônios responsáveis por esses elementos sensoriais leva à formação de um conjunto
celular. Um tempo depois, ao sentir o cheiro do mar novamente, esse estímulo pode acabar ativando o conjunto celular, resultando
na experiência da memória.

Um conjunto celular não é uma estrutura estática, mas sim uma rede dinâmica de neurônios que podem ser modificados e reativados ao
longo do tempo. A falta de estímulos ou a exposição a novas informações pode levar a uma modificação ou mesmo ao esquecimento de
partes da memória {...} 

<imagem demonstrando que diferentes conjuntos celulares representam memórias diferentes>


\section{Sono}

O sono é um processo biológico essencial que afeta muitos aspectos do funcionamento do cérebro, incluindo a consolidação da
memória e a plasticidade sináptica.

\section{Falta organizar melhor as seções}