# Neurônios biológicos

Neurônios recebem sinais dos dendritos, esses sinais são integrados no soma e, se for o caso, um potencial de ação é disparado
pelo axônio. O axônio carrega o sinal que é transmitido para os dendritos de outros neurônios através da sinapse.

Sinapses podem ser químicas (através de neurotransmissores), mais lentas mas que podem intesificar o sinal, ou elétricas, rápidas
mas que não podem mudar a amplitude do sinal. Sinapses podem ser excitatórias, que despolarizam a célula pós-sináptica, ou
inibitórias, que hiperpolarizam. Os neurotransmissores excitatórios mais comuns são: glutamato, dopamina e noradrenalina. Os
inibitórios são: GABA, glicina e serotonina.

O potencial de membrana é o potencial elétrico interior da célula em relação ao exterior. Em neurônios, o potencial de repouso é
de -70,15 mV. Quando ocorre um potencial de ação, canais de Na+ são abertos até que o potencial de reversão do sódio é alcançado,
em 38,43 mV. O potencial de membrana cai até chegar no potencial de reversão do K+ e hiperpolariza um pouco além do potencial de
repouso, para então voltar ao potencial de repouso. Diferentes neurônios têm potencial de repouso e de reversão de sódio e
potássio diferentes, entre mais ou menos 10 mV.

# Neurociência Computacional (?)

A neurociência computacional é um ramo da neurociência que utiliza modelos matemáticos e técnicas de simulação computacional para
investigar a estrutura e função do sistema nervoso. Essa área busca integrar conhecimentos teóricos e experimentais para entender
como os neurônios e redes neurais processam informações e realizam tarefas cognitivas, e aplicar esse conhecimento no
desenvolvimento de algoritmos e sistemas computacionais inspirados no cérebro.

# Redes Neurais de Disparos (RNDs)


# Modelos de Neurônios

Diferentes modelos de neurônios são utilizados em SNNs para simular o comportamento biológico dos neurônios. Alguns dos modelos
mais comuns incluem:

Modelo de Integrate-and-Fire: Neste modelo simplificado, o neurônio acumula potenciais sinápticos até atingir um limiar, quando
dispara um spike e reseta seu potencial de membrana.

Modelo de Hodgkin-Huxley: Um modelo mais complexo que descreve a dinâmica de
canais iônicos e a propagação do potencial de ação ao longo do axônio.

Modelo de Izhikevich: Um modelo que combina a simplicidade
do Integrate-and-Fire com a flexibilidade e biorealismo do Hodgkin-Huxley.

Nesse trabalho, foi utilizado o modelo XXXXX. (mais detalhes de pq foi utilizado esse modelo na metodologia)

# Modelos de Plasticidade

A plasticidade é um conceito chave na neurociência e se refere à capacidade dos neurônios de modificar suas sinapses e
comportamento em resposta a experiências e estímulos. Modelos de plasticidade incluem:

STP, STP, STDP, Homeostática, etc.

# Formação de Memórias por Conjuntos Celulares (Assembleias Celulares?)
 
A teoria de conjuntos celulares (do inglês, cell assemblies) propõe que a memória é armazenada na forma de conjuntos de neurônios
interconectados que são ativados simultaneamente. Esses conjuntos de neurônios são formados através da plasticidade sinápticao.
Quando um neurônio do conjunto é ativado após a formação da memória, isso leva à ativação de outros neurônios no conjunto, o que
permite a recuperação da memória.

<imagem demonstrando que diferentes conjuntos celulares representam memórias diferentes>

A formação e consolidação de memórias por conjuntos celulares podem ser influenciadas por fatores como a frequência e intensidade
dos estímulos, a sincronização dos disparos neuronais e a modulação neuromodulatória.

(falar como que em simulações simples os conjuntos celulares não são estáveis e degradam ou inflam com o tempo de simulação)
depois de um tempo)

# Sono

O sono é um processo biológico essencial que afeta muitos aspectos do funcionamento do cérebro, incluindo a consolidação da
memória e a plasticidade sináptica.