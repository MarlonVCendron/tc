(neurociência faz parte ou deriva da IA?)

(dentro do campo de IA existem várias )

A busca pela compreensão e reprodução das habilidades cognitivas e de aprendizado do cérebro humano tem sido um desafio constante nas áreas de neurociência computacional e
Inteligência Artificial (IA). É possível argumentar que as Redes Neurais Artificiais (RNA) são o mais próximo que já chegamos dessa reprodução; entretanto, as RNAs deixaram de lado
o realismo biológico em prol do aperfeiçoamento da IA. As Redes Neurais de Disparos (RND)¹ representam um avanço significativo em direção ao objetivo de compreender o cérebro
humano, uma vez que buscam emular o comportamento das Redes Neurais Biológicas (RNB) de forma mais realista do que as abordagens tradicionais.

As Redes Neurais Artificiais (RNA) convencionais são inspiradas no cérebro: neurônios disparam em determinadas frequências conforme os sinais recebidos de conexões com outros
neurônios através de sinapses plásticas, cuja força muda dinamicamente de acordo com o treinamento. Entretanto, as semelhanças com o cérebro terminam aqui. As RNAs tradicionais não
capturam a dinâmica interna dos neurônios biológicos, que disparam de maneiras complexas e distintas, e não apenas em uma frequência constante. A principal característica que torna
as RNAs capazes de aprender é seu método de retropropagação de erro, um método de treinamento que até pode existir em alguns casos no cérebro
\cite{lillicrapBackpropagationBrain2020} \cite{songCanBrainBackpropagation2020}, mas que é diferente da forma de de aprendizado local por plasticidade das RNBs
\cite{yamazakiSpikingNeuralNetworks2022}.

(Falar em quais problemas a RND é melhor)

(o que a RND traz de perspectiva de melhora) 

As RNDs são modelos muito mais próximos às RNBs que se comunicam por meio de impulsos elétricos discretos, chamados de disparos, e que aprendem por métodos realistas, como a
plasticidade nas sinapses. O grau de realismo biológico de uma RND depende de sua implementação, podendo empregar modelos de neurônios tão simples quanto a equação única do modelo
LIF, ou até modelos que simulam canais de íons, ramificações de dendritos, entre outros. As RNDs não só representam uma possível evolução das RNAs, mas também podem ser usadas para
seu propósito original: compreender o cérebro através da simulação.

No entanto, treinar RNDs continua sendo uma tarefa desafiadora, já que os algoritmos de aprendizado empregados nas RNAs tradicionais, além de não serem biologicamente realistas,
também não são diretamente aplicáveis às SNNs devido à natureza discreta dos disparos, que as torna não diferenciáveis.

# Sinapses e plasticidade Um dos métodos empregados para o aprendizado de RNDs é a plasticidade das sinapses. A plasticidade é a base do aprendizado e formação de memórias e ocorre
em diferentes escalas de tempo (limitações da plasticidade simples)

# Memória e aprendizado


# Sono

Notas rodapé:

¹ Do inglês Spiking Neural Networks.

