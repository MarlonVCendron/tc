\chapter{Introdução}

A busca pela compreensão e reprodução das habilidades cognitivas e de aprendizado do cérebro humano tem sido um desafio constante
nas áreas de neurociência computacional e Inteligência Artificial (IA). É possível argumentar que as Redes Neurais Artificiais
(RNA) são o mais próximo que já chegamos dessa reprodução; entretanto, as RNAs deixaram de lado o realismo biológico em prol do
aperfeiçoamento da IA \cite{}. As Redes Neurais de Disparos (RND)\footnote{Do inglês \textit{Spiking Neural Networks}.} representam um
avanço significativo em direção ao objetivo de compreender o cérebro humano, uma vez que buscam emular o comportamento das Redes
Neurais Biológicas (RNB) de forma mais realista do que as abordagens tradicionais.

As RNAs convencionais são inspiradas no cérebro: neurônios disparam em determinadas frequências conforme os sinais recebidos de
conexões com outros neurônios através de sinapses plásticas, cuja força muda dinamicamente de acordo com o treinamento.
Entretanto, as semelhanças com o cérebro terminam aqui. As RNAs tradicionais não capturam a dinâmica interna dos neurônios
biológicos, que disparam de maneiras complexas e distintas, e não apenas em uma determinada frequência. Outra diferença das RNAs
com sistemas biológicos é que as RNAs possuem um período de treinamento em que as sinapses são otimizadas, e um período em que não
há mais treinamento e as sinapses se tornam estáticas; enquanto nas RNBs as sinapses estão sempre se alterando conforme a
experiência, salvo nos raros casos em que há um período crítico de aprendizado durante a infância que é desativado quando o
indivíduo se torna adulto \cite{crepelRegression1982}. A principal característica que torna as RNAs capazes de aprender é seu
método de retropropagação de erro, um método de treinamento que até pode existir em alguns casos no cérebro
\cite{lillicrapBackpropagation2020; songCan2020}, mas que é diferente da forma de aprendizado local por plasticidade das
RNBs \cite{yamazakiSpiking2022}.

As RNDs são modelos muito mais próximos às RNBs que se comunicam por meio de impulsos elétricos discretos, chamados de disparos, e
que aprendem por métodos realistas, como a plasticidade nas sinapses. O grau de realismo biológico de uma RND depende de sua
implementação, podendo empregar modelos de neurônios tão simples quanto a equação única do modelo (apresentar)LIF, ou até modelos
que simulam canais de íons, ramificações de dendritos, entre outros. As RNDs não só representam uma possível evolução das RNAs,
como também são usadas para seu propósito original: compreender o cérebro através da simulação \cite{}.

// juntar os dois talvez

No entanto, treinar RNDs continua sendo uma tarefa desafiadora, já que os algoritmos de aprendizado empregados nas RNAs, além de
não serem biologicamente realistas, também não são diretamente aplicáveis às RNDs devido à natureza discreta dos disparos que as
torna não diferenciáveis, impedindo o cálculo de gradientes, parte fundamental no treinamento de RNAs \cite{}.

O principal método empregado para o aprendizado de RNDs é a plasticidade das sinapses. A plasticidade é a capacidade do cérebro de
se adaptar e reorganizar suas conexões neurais em resposta a novas informações, experiências ou estímulos; é a principal
propriedade por trás do aprendizado e da formação de memórias \cite{}. Uma das principais formas de plasticidade neural foi
primeiramente descrita por \cite{hebbOrganization1949}, chamada de plasticidade hebbiana, que pode resultar no fortalecimento ou
enfraquecimento das sinapses com base na ativação simultânea de neurônios conectados: caso o neurônio pós-sináptico dispare logo
após o neurônio pré-sináptico, significa que há uma correlação entre eles e a sinapse é fortalecida, caso contrário, a sinapse é
enfraquecida.

// em um parágrafo explicar conjuntos celulares, mais simplificado

// levar pra fundamentação teórica <<<
A plasticidade hebbiana pode ser sumarizada pela seguinte frase "Neurônios que disparam juntos, conectam-se juntos", assim, a
plasticidade dá origem a um fenômeno emergente no cérebro chamado de conjuntos celulares\footnote{Do inglês \textit{Cell
Assemblies}. Também traduzido como Assembleias Celulares.}, em que grupos de neurônios relacionados a um mesmo estímulo ou
processo acabam fortalecendo as conexões entre si. Um conjunto celular pode servir como uma forma de memória associativa
\cite{sakuraiMultiple2018} <<<Cell assemblies may be activated during memory recall, as evidenced by delay activity of neurons
during working memory tasks3,4 or during recognition of abstract items5,6,7>>>, em que a ativação de um dos neurônios do conjunto
acaba por ativar sincronamemente os demais neurônios do conjunto devido ao fortalecimento das conexões entre os neurônios do
conjunto. Tomando como exemplo a memória de uma viagem à praia: essa memória consiste em vários elementos, como o som das ondas, a
sensação de areia quente, o cheiro de água salgada e a visão de gaivotas etc. Cada um desses elementos sensoriais é processado em
diferentes áreas do cérebro e ativa diferentes grupos de neurônios. A ativação síncrona dos neurônios responsáveis por esses
elementos sensoriais leva à formação de um conjunto celular. Um tempo depois, ao sentir o cheiro do mar novamente, esse estímulo
pode acabar ativando o conjunto celular, resultando na experiência da memória.

Um conjunto celular não é uma estrutura estática, mas sim uma rede dinâmica de neurônios que podem ser modificados e reativados ao
longo do tempo. A falta de estímulos ou a exposição a novas informações pode levar a uma modificação ou mesmo ao esquecimento de
partes da memória. isso tá muito mal escrito
>>>

A plasticidade hebbiana, no entanto, não consegue gerar, por si só, conjuntos celulares estáveis quando simulada em uma RND; isso
ocorre pois a atividade neural continuamente modifica as sinapses, fazendo com que em pouco tempo quaisquer estímulos não
relacionados com a informação codificada no conjunto celular acabem alterando as sinapses e desfazendo o conjunto celular
\cite{gerstnerSpiking2002}.

Mas a plasticidade hebbiana não descreve toda a gama de diferentes modos com que a plasticidade se manifesta no cérebro, como é o
caso das  plasticidades heterossináptica, em que a ativação de neurônios causa mudanças em neurônios inativos, e homeostática, um
processo lento em que as sinapses se auto-regulam para garantir estabilidade. A plasticidade também depende do tipo de neurônio,
do tipo da conexão, do tempo de efeito das alterações (curto ou longo-prazo), entre outros fatores. A natureza do efeito da
plasticidade também varia muito, podendo depender da frequência de disparos, da diferença de potencial, do tempo dos disparos,
entre outros. Nas RNDs, assim como ocorre com os modelos de neurônios, os modelos de plasticidade também possuem uma ampla
variação em termos de plausibilidade biológica. Além disso, dependendo do modelo que se deseja utilizar, pode-se combinar
múltiplos modelos de plasticidade simultaneamente. Uma RND com plasticidade hebbiana junto de outras formas de plasticidade é
capaz de formar conjuntos celulares estáveis por horas \cite{zenkeDiverse2015}.

// Sono (encerrou uma ideia, começa a falar de outra)

falar sobre sono, o que é, pq é bom. ligar sono com memória (neurociência). Como ele influencia na retenção de memória

a nossa proposta é utilizar o sono


Neste contexto, o problema abordado neste trabalho consiste em explorar a retenção de memórias em uma RND. A principal hipótese a
ser avaliada neste trabalho é de que abordagens baseadas em simulações de fases do sono podem melhorar a estabilidade de conjuntos
celulares contribuindo para o processo de retenção de memórias.


-- Com a simulação de sono, ela aprenderia em menos tempo? Ela aprenderia mais padrões?
-- Como a rede se comporta se um novo conjunto de padrões é apresentado? os outros são esquecidos? e com o sono como ficaria?

\section{Objetivo geral}

O objetivo geral consiste em desenvolver simulações das diferentes fases do sono em RNPs, analisando como a consolidação e
retenção de memórias da rede pode ser afetada.


\section{Objetivos específicos}

Para melhor entendimento do objetivo geral, os seguintes objetivos específicos são propostos:

\begin{itemize}

  \item Investigar as características e propriedades das diferentes fases do sono para criar uma base sólida para a simulação das
das mesmas em uma RND.\@

  \item Estudar e selecionar o modelo de RND mais apropriado para a simulação das fases do sono, levando em consideração a
capacidade de representar a atividade neural durante o sono e a flexibilidade para incorporar diferentes mecanismos de
consolidação de memória.

  \item Sugerir e validar métodos para avaliar a consolidação e retenção de memórias na RND a fim de comparar a performance da
  RND com e sem a simulação do sono.

  \item  Analisar os resultados da simulação para identificar como a consolidação e retenção de memórias são afetadas pelas
diferentes fases do sono e pela atividade neural durante essas fases.

  \item Contribuir para o entendimento dos mecanismos subjacentes aos processos de aprendizado e memória no cérebro, assim como
discutir possíveis implicações e aplicações nos campos de neurociência computacional e inteligência artificial.

\end{itemize}




\section{Estrutura do trabalho}

O trabalho está estruturado da seguinte forma: No capítulo...