Objetivos específicos:

Implementar uma RND recorrente com neurônios LIF? e conexões sinápticas hebbianas e não hebbianas capaz de manter memórias de longo prazo.
Desenvolver uma simulação de sono inspirada nas observações de ondas cerebrais durante as diferentes fases do sono.
Avaliar a capacidade de memória da rede em relação aos padrões de imagens apresentados à ela.
Comparar a performance da RND com e sem a simulação do sono, a fim de verificar a hipótese de melhoria na capacidade de memória.
Analisar os resultados e discutir possíveis implicações e aplicações nos campos de neurociência computacional e inteligência artificial.
